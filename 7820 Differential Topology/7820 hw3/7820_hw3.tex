\documentclass[12pt,letterpaper]{article}

%--------Packages--------
\usepackage{amsmath, amsthm, amssymb}
\usepackage{xspace}
\usepackage{graphicx}
\usepackage{hhline}
\usepackage{amssymb}
\usepackage{array}
\usepackage{braket}
\usepackage{multicol}
\usepackage{mathtools}
\usepackage{enumerate}
\usepackage{delarray}
\usepackage{mathtools}
\usepackage{fullpage}
\usepackage{faktor} % For quotients
\usepackage{mathrsfs}

\usepackage[italicdiff]{physics} % For differentials
\usepackage{bbm} % For indicator

% \usepackage{quiver}
\usepackage[linguistics]{forest}




%--------Page Setup--------

\pagestyle{empty}%

\setlength{\hoffset}{-1.54cm}
\setlength{\voffset}{-1.54cm}

\setlength{\topmargin}{0pt}
\setlength{\headsep}{0pt}
\setlength{\headheight}{0pt}

\setlength{\oddsidemargin}{0pt}

\setlength{\textwidth}{195mm}
\setlength{\textheight}{250mm}


%--------Macros--------

\newcommand{\sub}{\subseteq}
\newcommand{\lcm}{\text{lcm}}
\newcommand{\mc}[1]{\mathcal{#1}}
\newcommand{\mf}[1]{\mathfrak{#1}}
\newcommand{\ms}[1]{\mathscr{#1}}
\newcommand{\sO}{\mathcal{O}}
\newcommand{\cyclic}[1]{\langle#1\rangle}
\newcommand{\units}[1]{#1 ^{\times}}
\newcommand{\la}{\langle}
\newcommand{\ra}{\rangle}
\newcommand{\lr}[1]{\left(#1\right)}
\newcommand{\lrvert}[1]{\left\lvert#1\right\rvert}
\DeclarePairedDelimiterX{\inp}[2]{\langle}{\rangle}{#1, #2}

%----Switch phi and varphi
% \let\temp\phi
% \let\phi\varphi
% \let\varphi\temp

\newcommand{\C}{\mathbb{C}}
\newcommand{\F}{\mathbb{F}}
\newcommand{\E}{\mathbb{E}}
\newcommand{\N}{\mathbb{N}\xspace}
\newcommand{\I}{\mathbb{I}\xspace}
\newcommand{\R}{\mathbb{R}\xspace}
\newcommand{\Z}{\mathbb{Z}\xspace}
\newcommand{\Q}{\mathbb{Q}\xspace}
\newcommand{\G}{\mathbb{G}\xspace}

\renewcommand{\H}{\mathcal{H}}
\newcommand{\M}{\mathcal{M}}

\DeclareMathOperator{\Spec}{Spec}
\DeclareMathOperator{\res}{res}
% \DeclareMathOperator{\Tr}{Tr}
\DeclareMathOperator{\ord}{ord}
\DeclareMathOperator{\Sym}{Sym}
% \DeclareMathOperator{\dv}{div}
\DeclareMathOperator{\alb}{alb}
\DeclareMathOperator{\img}{Im}
\DeclareMathOperator{\et}{et}
\DeclareMathOperator{\ck}{coker}
\DeclareMathOperator{\Reg}{Reg}
\DeclareMathOperator{\Cor}{Cor}
\DeclareMathOperator{\Ac}{at}
\DeclareMathOperator{\supp}{supp}
\DeclareMathOperator{\Hom}{Hom}
\DeclareMathOperator{\Pic}{Pic}
\DeclareMathOperator{\Gal}{Gal}
\DeclareMathOperator{\fc}{frac}
\DeclareMathOperator{\Ann}{Ann}
\DeclareMathOperator{\Mod}{Mod}
\DeclareMathOperator{\Cone}{Cone}
\DeclareMathOperator{\FI}{FI}
\DeclareMathOperator{\End}{End}
\DeclareMathOperator{\Alb}{Alb}
\DeclareMathOperator{\Ext}{Ext}
\DeclareMathOperator{\ab}{ab}
\DeclareMathOperator{\Jac}{Jac}
\DeclareMathOperator{\coker}{coker}
\DeclareMathOperator{\fr}{frac}
\DeclareMathOperator{\Int}{Int}
\let\Span\relax
\DeclareMathOperator{\Span}{Span}
\DeclareMathOperator{\Ran}{Ran}



%----Analysis
\newcommand{\summ}{\sum\limits}
% \newcommand{\norm}[1]{\left\lVert#1\right\rVert}
\newcommand{\thicc}{\bigg}
\newcommand{\eps}{\varepsilon}
\newcommand*\cls[1]{\overline{#1}}
\newcommand{\ind}{\mathbbm{1}}
\DeclareMathOperator{\sgn}{sgn}


%--------Theorem environments--------
\newtheorem{definition}{Definition}[]
\newtheorem{lemma}{Lemma}[]
\newtheorem{corollary}{Corollary}[]
\newtheorem{theorem}{Theorem}[]
\theoremstyle{remark}
\newtheorem*{claim}{Claim}


\newenvironment{solution}
{\begin{proof}[Solution]}
{\end{proof}}


\makeatletter
\newcommand{\thickhline}{%
    \noalign {\ifnum 0=`}\fi \hrule height 1pt
    \futurelet \reserved@a \@xhline
}
\newcolumntype{"}{@{\hskip\tabcolsep\vrule width 1pt\hskip\tabcolsep}}
\makeatother

% --------Problem environment--------
\setlength\parindent{0pt}
\setcounter{secnumdepth}{0}
\newcounter{partCounter}
\newcounter{homeworkProblemCounter}
\setcounter{homeworkProblemCounter}{1}


\newenvironment{homeworkProblem}[1][-1]{
    \ifnum#1>0
        \setcounter{homeworkProblemCounter}{#1}
    \fi
    \section{Problem \arabic{homeworkProblemCounter}}
    \setcounter{partCounter}{1}
    \stepcounter{homeworkProblemCounter}
}


%--------Metadata--------
\title{MATH 7820 Homework 3}
\author{James Harbour}

\begin{document}
\maketitle

\begin{homeworkProblem}
    Is the solution set to the system of equations 
    \[
        x^3 + y^{3}+ z^{3} = 1, \quad z=xy
    \]
    in $ \R^{3} $ a smooth manifold? Prove your answer.

    \begin{proof}
        Let $ S\sub \R^{3} $ be the solution set to the above system of equations. 
        Define $ F:\R^{3}\to\R^{2} $ by $(u,v) = F(x,y,z) = (x^{3}+y^{3}+z^{3}, xy-z) $. Then $ S = F^{-1}((1,0)) $. By the regular set theorem, it suffices to show that $ F^{-1}((1,0)) $ is a regular set. Hence, we must show that $ d_{p}F $ is surjective for all $ p\in S $, or equivalently, that $ \rank(J(F)_{p}) = 2$ for all $ p\in S $ where $ J(F) $ denotes the Jacobian of $ F $. We initially compute that
        \[
            J(F) = \begin{pmatrix}
                u_{x}&u_{y}& u_{z}\\ v_{x}&v_{y}&v_{z}
            \end{pmatrix} = 
            \begin{pmatrix}
                3x^{2} &3y^{2} & 3z^{2} \\ y & x & xy
            \end{pmatrix}
        \]

        Now, suppose $ p = (a,b,c)\in S $ is a critical point of $ F $, i.e. $ \rank(J(F)_{p}) <2 $. Then $ a^{3}+b^{3}+c^{3}=1 $ and $ c = ab $, so 
        \[
            J(F)_{p} = \mqty( 3a^{2} &3b^{2} & 3c^{2} \\ b & a & ab) = \mqty( 3a^{2} &3b^{2} & 3(ab)^{2} \\ b & a & ab). 
        \]
        If $ c = 0 $, then either $ a $ or $ b $ is $ 0 $, forcing the other to be $ 1 $, whence we obtain one of the following matrices,
        \[
            \begin{pmatrix}
                3&0&0\\0&1&0
            \end{pmatrix}, \quad
            \begin{pmatrix}
                0&3&0\\1&0&0
            \end{pmatrix},
        \]
        which are both of full rank contradicting the the criticality of $ p $. Thus, $ c\neq 0 $, so $ a,b\neq 0 $.
        Now by standard linear algebra, criticality of $ p $ is equivalent to the statement that all $ 2\times 2 $-minors of $ J(F)_{p} $ vanish. Thus, we have that 
        \[
            \det\begin{pmatrix}
                3a^{2}& 3b^{2} \\ b & a
            \end{pmatrix} = 0, 
            \qquad
            \det\begin{pmatrix}
                3a^{2}& 3a^{2}b^{2} \\ b & ab 
            \end{pmatrix} = 0,
            \qquad
            \det\begin{pmatrix}
                3b^{2}& 3a^{2}b^{2} \\ a & ab
            \end{pmatrix},
        \]
        whence we obtain the relations,
        \begin{equation*}
             \left.\begin{aligned}
                0&=3a^{3}-3b^{3}\\
                0&=3a^{3}b-3a^{2}b^{3}\\
                0&=3ab^{3}-3a^{3}b^{2}\\
            \end{aligned}\right\}\overset{a,b\neq0}{\implies}
            \left.\begin{aligned}
                0&=a^{3}-b^{3}\\
                0&=b-a^{2}\\
                0&=a-b^{2}\\
            \end{aligned}\right\}.
        \end{equation*} 
        Now, as $ b = a^2 $, it follows that $ 0 = a - b^2 = a-a^4$ whence $ a\neq 0 $ implies that $ a^{3}=1 $. As $ a\in \R $, it follows that $ a = 1 $ and consequently $  b=1 $,$ c=1 $. However, this contradicts the fact that $ p\in S $ as $3 =  1+1+1 = a^3 +b^3+c^3 \neq 1 $. Thus every point in $ S $ is regular, so $ F^{-1}((1,0)) $ is a regular level set whence by the regular set theorem $ S $ is a smooth manifold.
    \end{proof}
\end{homeworkProblem}


\begin{homeworkProblem}
    A $ C^{\infty} $ map $ f: N\to M $ is said to be \textit{transversal} to a submanifold $ S\sub M $ if for every $ p\in f^{-1}(S) $, 
    \[
        f_{*}(T_{p}N) + T_{f(p)}S = T_{f(p)}M.  
    \]
    The goal of this exercise is to prove the \textit{transversality theorem}: if a $ C^{\infty} $ map $ f:N\to M $ is transversal to a regular submanifold $ S $ of codimension $ k $ in $ M $, then $ f^{-1}(S) $ is a regular submanifold of codimension $ k $ in $ N $.\\

    When $ S $ consists of a single point $ c $, transversality of $ f $ to $ S $ simply means that $ f^{-1}(c) $ is a regular level set. Thus the transversality theorem is a generalization of the regular level set theorem. It is especially useful in giving conditions under which the intersection of two submanifolds is a submanifold.\\

    Let $ p\in f^{-1}(S)$ and $ (U, x^{1}, \ldots,x^{n}) $ be an adapted chart centered at $ f(p) $ for $ M $ relative to $ S $ such that $ U\cap S = Z(x^{m-k+1},\ldots, x^{m})$, the zero set of the functions $ x^{m-k+1},\ldots, x^{m} $. Define $ g:U\to\R^{k} $ to be the map 
    \[
        g = (x^{m-k+1},\ldots, x^{m}).
    \]
    
    \textbf{(a)}: Show that $ f^{-1}(U)\cap f^{-1}(S) = (g\circ f)^{-1}(0) $.

    \begin{proof}
        Observe that 
        \begin{align*}
            q\in (g\circ f)^{-1}(0)& \iff g(f(q)) = 0  \iff x^{i}(f(q)) = 0 \text{ for }i=m-k+1,\ldots,m \\
                                   &\iff f(q)\in Z(x^{m-k+1},\ldots, x^{m}) = U\cap S \iff q\in f^{-1}(U\cap S) = f^{-1}(U)\cap f^{-1}(S).
        \end{align*}
    \end{proof}

    \textbf{(b)}: Show that $ f^{-1}(U)\cap f^{-1}(S) $ is a regular level set of the function $ g\circ f : f^{-1}(U)\to \R^{k} $.

    \begin{proof}
        Fix $ p\in f^{-1}(S) \cap f^{-1}(U) = (g\circ f)^{-1}(0)$. We wish to show that $ p $ is a regular point for $ g\circ f $. Suppose that $ a\in T_{0}\R^{k} $. Then, noting that $ \dd{g}_{f(p)} $ is surjective, there exists a $ w\in T_{f(p)}M $ such that $ \dd{g}_{f(p)}(w)=a $. Then, by transversality of $ f $ with respect to $ S $, there exist $ u\in T_{p}N $, $ v\in T_{f(p)}S $ such that $ w = \dd{f}_{p}(u) + v = w $. Now, note that $ g(U\cap S) = 0 $, so it follows that $ \dd{g}_{f(p)}(T_{f(p)}(S)) = 0 $. Hence, we compute
        \[
            a = \dd{g}_{f(p)}(w) = \dd{g}_{f(p)}(\dd{f}_{p}(u)+v) = \dd{g\circ f}_{p}(u) + \dd{g}_{f(p)}(v) = \dd{g\circ f}_{p}(u),
        \]
        so $ p $ is a regular point for $ g\circ f $ as its differential is surjective.
    \end{proof}

    \textbf{(c)}: Prove the transversality theorem.
    
    \begin{proof}
        By the regular level set theorem, $ f^{-1}(U)\cap f^{-1} (S)$ is a codimension $ k $ submanifold of $ N $. For $ q\in S$, choose a chart $ (V_{q},\phi_{q}) $ adapted to $ q $. Then $ f^{-1}(V_{q}\cap S) \sub f^{-1}(S) $ and $ f^{-1}(S) = \bigcup_{q\in S}f^{-1}(V_{q}\cap S)$, whence $ f^{-1}(S) $ is also a codimension $ k $ submanifold of $ N $, as desired.
    \end{proof}
\end{homeworkProblem}


\begin{homeworkProblem}
    \textbf{(a)}: Consider the "height map" $ h:S^{2}\to\R $. Here $ S^{2 } $ is the unit sphere in $ \R^{3} $ and $ h(x,y,z) = z $. Find the critical points and critical values for this map.

    \begin{proof}
        First suppose $ p=(a,b,c)\in S^{2} $ with $ c>0 $. Consider the chart $ (U,\phi) $ on $ S^{2} $ given by $ U = \{(x,y,z)\in S^{2}: z>0\} $ and $ \phi(x,y,z) = (x,y) $. Let $ \widetilde{h}:\phi(U)\to \R $ be the coordinate representation of $ h $ with respect to this chart. Then $ \widetilde{h}(x,y) = (h\circ \phi^{-1})(x,y) = h(x,y,\sqrt{1-x^{2}-y^{2}}) = \sqrt{1-x^{2}-y^{2}}$, whence 
        \[
            \dd{\widetilde{h}}_p = \mqty(\frac{-x}{\sqrt{1-x^{2}-y^{2}}} & \frac{-y}{\sqrt{1-x^{2}-y^{2}}}),
        \]
        which has rank $ 0 $ if and only if $ x,y = 0 $, whence $ p=(0,0,1) $. Thus $ p=(0,0,1) $ is the only critical point of $ h $ in $ U $ and has critical value $ 1 $. \\

        If $ p=(a,b,c)\in S_{2} $ with $ c<0 $, then we have a similar situation except that $ \widetilde{h}(x,y) = -\sqrt{1-x^{2-y^{2}}} $ and our chart is $ (V,\psi) $ where $ V = \{(x,y,z)\in S^{2}: z<0\} $ and $ \psi^{-1}(x,y) = (x,y,-\sqrt{1-x^{2}-y^{2}}) $. Thus, again $ \dd{\widetilde{h}}_{p} $ has rank $ 0 $ if and only if $ x,y=0 $. So, in this case $ p = \psi^{-1}(0,0) = (0,0,-1) $ is the only critical point of $ h $ in $ V $ and has critical value $ -1 $.\\

        Now must check points on the equator $ E = \{(x,y,z)\in S^{2}: z=0\} $. Consider points $ p $ in the chart $ (W,\rho) $ with $ W = \{(x,y,z)\in S^{2}: y>0\} $ and $ \rho(x,y,z)= (x,z) $. We compute that the coordinate representation of $ h $ is then given by $ \widetilde{h}(x,z) = z $, whence $ \dd{F}_{p}= \mqty(0&1) $ for all $ p\in W$, so no points in $ W $ can be critical. Similarly, no points in $ W ' = \{(x,y,z)\in S^{2}: y<0\} $ can be critical either (identical calulation).\\

        Thus, it remains to check the points $ (1,0,0) $ and $ (-1,0,0) $. Consider the chart $ (K,\gamma) $ given by $ K= \{x>0\} $ and $ \gamma(x,y,z) = (y,z) $. Then again, $ \widetilde{h}(y,z) = z $, whence no points in $ K $ can be critical. Similarly, no points in $ K '= \{(x,y,z)\in S^2 : x>0\} $ can be critical either (by the same calculation).
    \end{proof}

    \textbf{(b)}: Show that any map $ f:S^{2}\to \R $ has at least two critical points. Generalize this proof from $ S^{2} $ to any $ n $-dimensional compact manifold.

    \begin{proof}

        
        Since $ f $ is continuous and $ S^{2} $ is compact, it follows that there exist $ p,q\in S^{2} $ such that $ f(p)\leq f(x) \leq f(q) $ for all $ x\in S^{2} $ and $ p\neq q $. Thus $ p,q $ are global minima/maxima of the function $ f $. Choose a chart $ (U,\phi) $ on $ S^{2} $ such that $ p,q\in U $. Then the function $ \widetilde{f}:\phi(U)\to \R $ given by $ \widetilde{f}= f\circ \phi^{-1} $ is smooth as a function from $ \R^2\to R $ and has global minima/maxima $ p,q $. Thus, by calculus 3, 
         \[ 
            \pdv{\widetilde{f}}{x}(p) = \pdv{\widetilde{f}}{y} (p) = \pdv{\widetilde{f}}{x} (q) = \pdv{\widetilde{f}}{y} (q) = 0,
        \] 
        whence $ \dd{f}_{p} $ has rank $ 0 $ at both $ p $ and $ q $, so $ p $ and $ q $ are critical points of $ f $.\\

        Now suppose that $ F:M\to \R $ is a smooth map from an $ n $-dimensional compact manifold $ M $. Again, by continuity, there exist $ p,q\in M $ such that $ f(p)\leq f(x)\leq f(q) $ for all $ x\in M $ and $ p\neq q $. Again, choose a chart $ (U,\phi) $ containing both $ p $ and $ q $. Then $ \widetilde{F} = F\circ \phi^{-1} $ is a smooth function from $ \phi(U)\sub \R^n $ to $ \R $ with global minima/maxima $ p,q $. Again, for $ i=1,\ldots, n $, it follows that
        \[
            \pdv{\widetilde{F}}{x_{i}} (p) = \pdv{\widetilde{F}}{x_{i}}(q) =0,
        \]
        whence the jacobian of $ \widetilde{F} $ at $ p $ and at $ q $ has rank $ 0 $, making $ p,q $ critical points of $ F $.

    \end{proof}
\end{homeworkProblem}


\begin{homeworkProblem}
    Consider a submanifold $ M^{n}\sub\R^{k} $ and let $ TM\sub \R^{k}\times\R^{k} $ be the set of all pairs $ (x,v) $ where $ x $ is a point in $ M $ and $ v\in T_{x}M $. Show that $ TM $ is a smooth $ 2n $-dimensional submanifold of $ \R^{2k} $.

    \begin{proof}
        Fix $ (x,v)\in TM $ and let $ (U,\phi) $ be a chart on $ \R^{k} $ adapted to $ M $ about $ p $. Set $ V = U\cap M $, $ \widetilde{U} = U\times \R^{k} $, and $ \widetilde{V} = \bigsqcup_{y\in V}(\{y\}\times T_{y}M) \sub TM$. Let $ \widetilde{\phi}:\widetilde{U}\to \R^k\times \R^k$ be given by $ \widetilde{\phi}(y,w) = (\phi
        (y), \dd{\phi}_{y}(w)) $. Note that, after identifying $ T_y M \cong R^n\sub \R^k$ for each $ y\in U $, we have that $ \widetilde{V} = \widetilde{U}\cap TM $. . As $ (U,\phi) $ is adapted to $ M $, by definition $ \phi(U) = \phi(U\cap M)\times\{0\} \sub \R^n\times\R^{k-n}$. Now compute that
        \begin{align*}
            \widetilde{\phi} (\widetilde{U}) &= \{ (\phi(p), \dd{\phi}_{p}(v)) : p\in V, v\in\R^n\} \\
            &= \{(\phi\vert_{V}, 0, \ldots, 0, \dd{\phi\vert_{V}}(v), 0, \ldots, 0)\in \R^k\times\R^k: p\in V, v\in T_{p}M \} \\
            &= \widetilde{\phi}(\widetilde{U}\cap TM)\times \R^{2(k-n)}.
        \end{align*}
        Hence, it suffices to show that if $ (U,\phi), (V,\psi)$ are two charts on $ \R^k $ adapted to $ M $, that the transition maps $ \widetilde{\phi}\vert\circ \widetilde{\psi}^{-1}$ is smooth (the other direction would follow since the charts chosen are arbitrary). Let $ p\in \psi(U\cap V), v\in \R^{k} $. Then 
        \[
            \widetilde{\phi}\vert\circ \widetilde{\psi}^{-1} (p,v) = (\phi\circ \psi^{-1}(p), \dd{\phi}_{\psi^{-1}(p)}\circ \dd{\psi}^{-1}_{p}(v)) = (\phi\circ \psi^{-1}(p), \dd{(\phi\circ \psi^{-1})}_{p}(v)),
        \]
        which is smooth as the jacobian smoothly depends upon $ p,v $.
    \end{proof}
\end{homeworkProblem}




\end{document}
