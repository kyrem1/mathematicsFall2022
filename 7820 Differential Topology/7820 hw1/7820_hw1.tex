\documentclass[12pt,letterpaper]{article}

%--------Packages--------
\usepackage{amsmath, amsthm, amssymb}
\usepackage{xspace}
\usepackage{graphicx}
\usepackage{hhline}
\usepackage{amssymb}
\usepackage{array}
\usepackage{braket}
\usepackage{multicol}
\usepackage{mathtools}
\usepackage{enumerate}
\usepackage{delarray}
\usepackage{mathtools}
\usepackage{fullpage}
\usepackage{faktor} % For quotients
\usepackage{mathrsfs}

\usepackage[italicdiff]{physics} % For differentials
\usepackage{bbm} % For indicator

% \usepackage{quiver}
\usepackage[linguistics]{forest}




%--------Page Setup--------

\pagestyle{empty}%

\setlength{\hoffset}{-1.54cm}
\setlength{\voffset}{-1.54cm}

\setlength{\topmargin}{0pt}
\setlength{\headsep}{0pt}
\setlength{\headheight}{0pt}

\setlength{\oddsidemargin}{0pt}

\setlength{\textwidth}{195mm}
\setlength{\textheight}{250mm}


%--------Macros--------

\newcommand{\sub}{\subseteq}
\newcommand{\lcm}{\text{lcm}}
\newcommand{\mc}[1]{\mathcal{#1}}
\newcommand{\mf}[1]{\mathfrak{#1}}
\newcommand{\ms}[1]{\mathscr{#1}}
\newcommand{\sO}{\mathcal{O}}
\newcommand{\cyclic}[1]{\langle#1\rangle}
\newcommand{\units}[1]{#1 ^{\times}}
\newcommand{\la}{\langle}
\newcommand{\ra}{\rangle}
\newcommand{\lr}[1]{\left(#1\right)}
\newcommand{\lrvert}[1]{\left\lvert#1\right\rvert}

\DeclarePairedDelimiterX{\inp}[2]{\langle}{\rangle}{#1, #2}

%----Switch phi and varphi
% \let\temp\phi
% \let\phi\varphi
% \let\varphi\temp

\newcommand{\C}{\mathbb{C}}
\newcommand{\F}{\mathbb{F}}
\newcommand{\E}{\mathbb{E}}
\newcommand{\N}{\mathbb{N}\xspace}
\newcommand{\I}{\mathbb{I}\xspace}
\newcommand{\R}{\mathbb{R}\xspace}
\newcommand{\Z}{\mathbb{Z}\xspace}
\newcommand{\Q}{\mathbb{Q}\xspace}
\newcommand{\G}{\mathbb{G}\xspace}

\renewcommand{\H}{\mathcal{H}}
\newcommand{\M}{\mathcal{M}}

\DeclareMathOperator{\Spec}{Spec}
\DeclareMathOperator{\res}{res}
% \DeclareMathOperator{\Tr}{Tr}
\DeclareMathOperator{\ord}{ord}
\DeclareMathOperator{\Sym}{Sym}
% \DeclareMathOperator{\dv}{div}
\DeclareMathOperator{\alb}{alb}
\DeclareMathOperator{\img}{Im}
\DeclareMathOperator{\et}{et}
\DeclareMathOperator{\ck}{coker}
\DeclareMathOperator{\Reg}{Reg}
\DeclareMathOperator{\Cor}{Cor}
\DeclareMathOperator{\Ac}{at}
\DeclareMathOperator{\supp}{supp}
\DeclareMathOperator{\Hom}{Hom}
\DeclareMathOperator{\Pic}{Pic}
\DeclareMathOperator{\Gal}{Gal}
\DeclareMathOperator{\fc}{frac}
\DeclareMathOperator{\Ann}{Ann}
\DeclareMathOperator{\Mod}{Mod}
\DeclareMathOperator{\Cone}{Cone}
\DeclareMathOperator{\FI}{FI}
\DeclareMathOperator{\End}{End}
\DeclareMathOperator{\Alb}{Alb}
\DeclareMathOperator{\Ext}{Ext}
\DeclareMathOperator{\ab}{ab}
\DeclareMathOperator{\Jac}{Jac}
\DeclareMathOperator{\coker}{coker}
\DeclareMathOperator{\fr}{frac}
\DeclareMathOperator{\Int}{Int}
\let\Span\relax
\DeclareMathOperator{\Span}{Span}
\DeclareMathOperator{\Ran}{Ran}



%----Analysis
\newcommand{\summ}{\sum\limits}
% \newcommand{\norm}[1]{\left\lVert#1\right\rVert}
\newcommand{\thicc}{\bigg}
\newcommand{\eps}{\varepsilon}
\newcommand*\cls[1]{\overline{#1}}
\newcommand{\ind}{\mathbbm{1}}
\DeclareMathOperator{\sgn}{sgn}


%--------Theorem environments--------
\newtheorem{definition}{Definition}[]
\newtheorem{lemma}{Lemma}[]
\newtheorem{corollary}{Corollary}[]
\newtheorem{theorem}{Theorem}[]
\theoremstyle{remark}
\newtheorem*{claim}{Claim}


\newenvironment{solution}
{\begin{proof}[Solution]}
{\end{proof}}


\makeatletter
\newcommand{\thickhline}{%
    \noalign {\ifnum 0=`}\fi \hrule height 1pt
    \futurelet \reserved@a \@xhline
}
\newcolumntype{"}{@{\hskip\tabcolsep\vrule width 1pt\hskip\tabcolsep}}
\makeatother

% --------Problem environment--------
\setlength\parindent{0pt}
\setcounter{secnumdepth}{0}
\newcounter{partCounter}
\newcounter{homeworkProblemCounter}
\setcounter{homeworkProblemCounter}{1}


\newenvironment{homeworkProblem}[1][-1]{
    \ifnum#1>0
        \setcounter{homeworkProblemCounter}{#1}
    \fi
    \section{Problem \arabic{homeworkProblemCounter}}
    \setcounter{partCounter}{1}
    \stepcounter{homeworkProblemCounter}
}


%--------Metadata--------
\title{MATH 7820 Homework 1}
\author{James Harbour}

\begin{document}
\maketitle


\begin{homeworkProblem}[2]
  Let $\{(U_\alpha,\phi_\alpha)\}$ and $\{(V_\beta,\psi_\beta)\}$ be atlases for smooth manifolds $M$ and $N$ of dimensions $m$ and $n$ respectively. Show that the collection $\{(U_\alpha\times V_\beta,\phi_\alpha\times \psi_\beta)\}$ of charts is an atlas on $M\times N$. Therefore, $M\times N$ is a smooth manifold of dimension $m+n$.

  \begin{proof}
    For $p\in M$ and $q\in N$, there exist $\alpha,\beta$ such that $p\in U_\alpha$ and $q\in V_\beta$ whence $(p,q)\in U_\alpha\times V_\beta$. Thus $M\times N = \bigcup_{\alpha,\beta}U_\alpha\times V_\beta$. \\

    For each $\alpha,\beta$, as $\phi_\alpha$ and $\psi_\beta$ are homeomorphisms onto $\phi_\alpha(U_\alpha)$ and $\psi_\beta(V_\beta)$ respectively, it follows that $\phi_\alpha\times \psi_\beta$ is a homeomorphism onto $\phi_\alpha\times\psi_\beta(U_\alpha\times V_\beta) = \phi_\alpha(U_\alpha)\times \psi(V_\beta)$. \\

    Fix $\alpha,\beta$. Then we compute the transition maps
    \begin{align*}
      (\phi_{\alpha'}\times\psi_{\beta'})\circ(\phi_{\alpha}\times\psi_{\beta})^{-1} &= (\phi_{\alpha'}\times\psi_{\beta'}) \circ(\phi_{\alpha}^{-1}\times\psi_{\beta}^{-1}) = (\phi_{\alpha'}\circ\phi_{\alpha}^{-1})\times(\psi_{\beta'}\circ\psi_{\beta}^{-1}) \\
      (\phi_{\alpha}\times\psi_{\beta}) \circ (\phi_{\alpha'}\times\psi_{\beta'})^{-1} &=(\phi_{\alpha}\times\psi_{\beta}) \circ(\phi_{\alpha'}^{-1}\times\psi_{\beta'}^{-1}) = (\phi_{\alpha}\circ\phi_{\alpha'}^{-1})\times(\psi_{\beta}\circ\psi_{\beta'}^{-1})
    \end{align*}
    which are both smooth as $\phi_{\alpha}\circ\phi_{\alpha'}^{-1}$, $\phi_{\alpha'}\circ\phi_{\alpha}^{-1}$,  $\psi_{\beta}\circ\psi_{\beta'}^{-1}$, $\psi_{\beta'}\circ\psi_{\beta}^{-1}$ are smooth. Thus $\{(U_\alpha\times V_\beta,\phi_\alpha\times \psi_\beta)\}$ is an atlas on $M\times N$.
  \end{proof}
\end{homeworkProblem}

\end{document}
