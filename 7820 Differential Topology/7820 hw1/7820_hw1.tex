\documentclass[12pt,letterpaper]{article}

%--------Packages--------
\usepackage{amsmath, amsthm, amssymb}
\usepackage{xspace}
\usepackage{graphicx}
\usepackage{hhline}
\usepackage{amssymb}
\usepackage{array}
\usepackage{braket}
\usepackage{multicol}
\usepackage{mathtools}
\usepackage{enumerate}
\usepackage{delarray}
\usepackage{mathtools}
\usepackage{fullpage}
\usepackage{faktor} % For quotients
\usepackage{mathrsfs}

\usepackage[italicdiff]{physics} % For differentials
\usepackage{bbm} % For indicator

% \usepackage{quiver}
\usepackage[linguistics]{forest}




%--------Page Setup--------

\pagestyle{empty}%

\setlength{\hoffset}{-1.54cm}
\setlength{\voffset}{-1.54cm}

\setlength{\topmargin}{0pt}
\setlength{\headsep}{0pt}
\setlength{\headheight}{0pt}

\setlength{\oddsidemargin}{0pt}

\setlength{\textwidth}{195mm}
\setlength{\textheight}{250mm}


%--------Macros--------

\newcommand{\sub}{\subseteq}
\newcommand{\lcm}{\text{lcm}}
\newcommand{\mc}[1]{\mathcal{#1}}
\newcommand{\mf}[1]{\mathfrak{#1}}
\newcommand{\ms}[1]{\mathscr{#1}}
\newcommand{\sO}{\mathcal{O}}
\newcommand{\cyclic}[1]{\langle#1\rangle}
\newcommand{\units}[1]{#1 ^{\times}}
\newcommand{\la}{\langle}
\newcommand{\ra}{\rangle}
\newcommand{\lr}[1]{\left(#1\right)}
\newcommand{\lrvert}[1]{\left\lvert#1\right\rvert}

\DeclarePairedDelimiterX{\inp}[2]{\langle}{\rangle}{#1, #2}

%----Switch phi and varphi
% \let\temp\phi
% \let\phi\varphi
% \let\varphi\temp

\newcommand{\C}{\mathbb{C}}
\newcommand{\F}{\mathbb{F}}
\newcommand{\E}{\mathbb{E}}
\newcommand{\N}{\mathbb{N}\xspace}
\newcommand{\I}{\mathbb{I}\xspace}
\newcommand{\R}{\mathbb{R}\xspace}
\newcommand{\Z}{\mathbb{Z}\xspace}
\newcommand{\Q}{\mathbb{Q}\xspace}
\newcommand{\G}{\mathbb{G}\xspace}

\renewcommand{\H}{\mathcal{H}}
\newcommand{\M}{\mathcal{M}}

\DeclareMathOperator{\Spec}{Spec}
\DeclareMathOperator{\GL}{GL}
\DeclareMathOperator{\res}{res}
% \DeclareMathOperator{\Tr}{Tr}
\DeclareMathOperator{\ord}{ord}
\DeclareMathOperator{\Sym}{Sym}
% \DeclareMathOperator{\dv}{div}
\DeclareMathOperator{\alb}{alb}
\DeclareMathOperator{\img}{Im}
\DeclareMathOperator{\et}{et}
\DeclareMathOperator{\ck}{coker}
\DeclareMathOperator{\Reg}{Reg}
\DeclareMathOperator{\Cor}{Cor}
\DeclareMathOperator{\Ac}{at}
\DeclareMathOperator{\supp}{supp}
\DeclareMathOperator{\Hom}{Hom}
\DeclareMathOperator{\Pic}{Pic}
\DeclareMathOperator{\Gal}{Gal}
\DeclareMathOperator{\fc}{frac}
\DeclareMathOperator{\Ann}{Ann}
\DeclareMathOperator{\Mod}{Mod}
\DeclareMathOperator{\Cone}{Cone}
\DeclareMathOperator{\FI}{FI}
\DeclareMathOperator{\End}{End}
\DeclareMathOperator{\Alb}{Alb}
\DeclareMathOperator{\Ext}{Ext}
\DeclareMathOperator{\ab}{ab}
\DeclareMathOperator{\Jac}{Jac}
\DeclareMathOperator{\coker}{coker}
\DeclareMathOperator{\fr}{frac}
\DeclareMathOperator{\Int}{Int}
\let\Span\relax
\DeclareMathOperator{\Span}{Span}
\DeclareMathOperator{\Ran}{Ran}



%----Analysis
\newcommand{\summ}{\sum\limits}
% \newcommand{\norm}[1]{\left\lVert#1\right\rVert}
\newcommand{\thicc}{\bigg}
\newcommand{\eps}{\varepsilon}
\newcommand*\cls[1]{\overline{#1}}
\newcommand{\ind}{\mathbbm{1}}
\DeclareMathOperator{\sgn}{sgn}


%--------Theorem environments--------
\newtheorem{definition}{Definition}[]
\newtheorem{lemma}{Lemma}[]
\newtheorem{corollary}{Corollary}[]
\newtheorem{theorem}{Theorem}[]
\theoremstyle{remark}
\newtheorem*{claim}{Claim}


\newenvironment{solution}
{\begin{proof}[Solution]}
{\end{proof}}


\makeatletter
\newcommand{\thickhline}{%
    \noalign {\ifnum 0=`}\fi \hrule height 1pt
    \futurelet \reserved@a \@xhline
}
\newcolumntype{"}{@{\hskip\tabcolsep\vrule width 1pt\hskip\tabcolsep}}
\makeatother

% --------Problem environment--------
\setlength\parindent{0pt}
\setcounter{secnumdepth}{0}
\newcounter{partCounter}
\newcounter{homeworkProblemCounter}
\setcounter{homeworkProblemCounter}{1}


\newenvironment{homeworkProblem}[1][-1]{
    \ifnum#1>0
        \setcounter{homeworkProblemCounter}{#1}
    \fi
    \section{Problem \arabic{homeworkProblemCounter}}
    \setcounter{partCounter}{1}
    \stepcounter{homeworkProblemCounter}
}


%--------Metadata--------
\title{MATH 7820 Homework 1}
\author{James Harbour}

\begin{document}
\maketitle

\begin{homeworkProblem}
  Let $S^2$ be the unit sphere in $\R^3$. Define in $S^2$ the six charts cooresponding to the six hemispheres:
  \[
    U_1 = \{(x,y,z)\in S^2: x>0\},\, \phi_1(x,y,z) = (y,z),
  \]
  and the analogous
  \[
    U_2=\{x<0\},\, U_3=\{y>0\},\, U_4 = \{y<0\},\, U_5 = \{z>0\},\, U_6 = \{z<0\}.
  \]
  Describe the domain $\phi_4(U_1\cap U_4)$ of $\phi_1\circ\phi_4^{-1}$ and show that $\phi_1\circ\phi_4^{-1}$ is smooth on its domain.

  \begin{solution}
    We compute that
    \[
      \phi_{4} (U_1\cap U_4) = \{\phi_4(x,y,z): x^2 +y^2 +z^2 = 1,\, x>0,\, y<0\} = \{(x,z)\in \R^2:x>0,\, x^2+z^2 < 1\},
    \]
    which is a semicircular open disk in the right half $(x,z)$-plane. If $(x,z)\in \phi_{4} (U_1\cap U_4)$, then $\phi_4^{-1}(x,z) = (x,\sqrt{1-x^2-z^2},z)$ whence $\phi_1\circ\phi_4^{-1}(x,z) = (x,\sqrt{1-x^2-z^2})$. This function is smooth on $\phi_{4} (U_1\cap U_4)$ since $x^2+z^2<1$.
  \end{solution}
\end{homeworkProblem}


\begin{homeworkProblem}[2]
  Let $\{(U_\alpha,\phi_\alpha)\}$ and $\{(V_\beta,\psi_\beta)\}$ be atlases for smooth manifolds $M$ and $N$ of dimensions $m$ and $n$ respectively. Show that the collection $\{(U_\alpha\times V_\beta,\phi_\alpha\times \psi_\beta)\}$ of charts is an atlas on $M\times N$. Therefore, $M\times N$ is a smooth manifold of dimension $m+n$.

  \begin{proof}
    For $p\in M$ and $q\in N$, there exist $\alpha,\beta$ such that $p\in U_\alpha$ and $q\in V_\beta$ whence $(p,q)\in U_\alpha\times V_\beta$. Thus $M\times N = \bigcup_{\alpha,\beta}U_\alpha\times V_\beta$. \\

    For each $\alpha,\beta$, as $\phi_\alpha$ and $\psi_\beta$ are homeomorphisms onto $\phi_\alpha(U_\alpha)$ and $\psi_\beta(V_\beta)$ respectively, it follows that $\phi_\alpha\times \psi_\beta$ is a homeomorphism onto $\phi_\alpha\times\psi_\beta(U_\alpha\times V_\beta) = \phi_\alpha(U_\alpha)\times \psi(V_\beta)$. \\

    Fix $\alpha,\beta$. Then we compute the transition maps
    \begin{align*}
      (\phi_{\alpha'}\times\psi_{\beta'})\circ(\phi_{\alpha}\times\psi_{\beta})^{-1} &= (\phi_{\alpha'}\times\psi_{\beta'}) \circ(\phi_{\alpha}^{-1}\times\psi_{\beta}^{-1}) = (\phi_{\alpha'}\circ\phi_{\alpha}^{-1})\times(\psi_{\beta'}\circ\psi_{\beta}^{-1}) \\
      (\phi_{\alpha}\times\psi_{\beta}) \circ (\phi_{\alpha'}\times\psi_{\beta'})^{-1} &=(\phi_{\alpha}\times\psi_{\beta}) \circ(\phi_{\alpha'}^{-1}\times\psi_{\beta'}^{-1}) = (\phi_{\alpha}\circ\phi_{\alpha'}^{-1})\times(\psi_{\beta}\circ\psi_{\beta'}^{-1})
    \end{align*}
    which are both smooth as $\phi_{\alpha}\circ\phi_{\alpha'}^{-1}$, $\phi_{\alpha'}\circ\phi_{\alpha}^{-1}$,  $\psi_{\beta}\circ\psi_{\beta'}^{-1}$, $\psi_{\beta'}\circ\psi_{\beta}^{-1}$ are smooth. Thus $\{(U_\alpha\times V_\beta,\phi_\alpha\times \psi_\beta)\}$ is an atlas on $M\times N$.
  \end{proof}
\end{homeworkProblem}

\begin{homeworkProblem}
  Let $V$ be a finite-dimensional vector space over $\R$. Relative to and ordered basis $e = (e_1,\ldots,e_n)$ for $V$, an element $L\in\GL(V)$ is represented by a matrix $[a_j^i]$ defined by
  \[
    L(e_j) = \sum_{i}a_{j}^i e_i.
  \]
  The map
  \[
    \phi_e:\GL(V)\to \GL(n,\R),\text{ }L\mapsto [a_j^i],
  \]
  is a bijection with an open subset of $\R^{n\times n}$ that makes $\GL(V)$ into a smooth manifold, which we denote temporarily by $\GL(V)_e$. If $\GL(V)_u$ is the manifold structure induced from another ordered basis $u = (u_1,\ldots, u_n)$ for $V$, show that $\GL(V)_e$ is the same as $\GL(V)_u$. \\

  \begin{proof}
    To show that $\GL(V)_e$ is the same as $\GL(V)_u$, it suffices to show that the charts $(\GL(V),\phi_e)$ and $(\GL(V),\phi_u)$ have the same maximal atlas, as then the smooth manifold structures that they define would be the same. Hence, it suffices to show that these charts are in fact compatible.\\

    For an ordered basis $\beta$ of $\R^n$ and $T\in \GL(V)$, we write $[T]_{\beta}\in\GL(n,\R)$ to denote the matrix representation of $T$ with respect to the basis $\beta$. If $\beta$,$\gamma$ are two ordered bases of $R^n$, we write $[I]_\gamma^{\beta}$ to denote the change-of-basis matrix from $\beta$ to $\gamma$.\\

    Let $A\in \GL(n,\R)$. Then there exists a unique $T\in \GL(V)$ such that $A = [T]_e = \phi_e(T)$, so $\phi_e^{-1}(A) = T$. Then
    \[
      \phi_u\circ\phi_e^{-1}(A) = \phi_u(T) = [T]_u = [I]_u^e[T]_e = [I]_u^e A,
    \]
    which is a polynomial function in the coordinates of $\GL(n,\R)$ and thus smooth. In a similar vein, if $B\in \GL(n,\R)$, there is a unique $S\in \GL(V)$ such that $B = [S]_u = \phi_u(S)$, so $\phi_u^{-1}(B) = S$. Then
    \[
      \phi_e\circ\phi_u^{-1}(B) = \phi_e(S) = [S]_e = [I]_e^u[S]_u = [I]_e^u B,
    \]
    which is also smooth.
  \end{proof}
\end{homeworkProblem}

\begin{homeworkProblem}[4]
  Find all points in $\R^3$ in a neighborhood of which the functions $x,x^2+y^2+z^2-1,z$ can serve as a local coordinate system.

  \begin{proof}
    Let $F:\R^3\to\R^3$ be given by $F(x,y,z) = (x,x^2+y^2+z^2-1,z)$. By Corollary 6.27, it suffices to find all $p\in \R^3$ for which the Jacobian matrix $[\partial F^i/\partial x^j (p)]_{i,j}$ is nonsingular. Hence, we compute
    \[
      \det [\frac{\partial F^i}{\partial x^j}(p)] = \det \begin{pmatrix}
        1 & 0 & 0\\ 2x & 2y & 2z \\ 0 & 0 & 1
    \end{pmatrix} = 2x,
    \]
    thus the Jacobian matrix for $F$ is nonsingular for all $(x,y,z)\in \R^3$ such that $x\neq 0$. Taking any point $p$ such that this holds, we can find always small enough open neighborhood of $p$ not intersecting the $x$-axis, thus giving a local coordinate system.
  \end{proof}
\end{homeworkProblem}

\end{document}
