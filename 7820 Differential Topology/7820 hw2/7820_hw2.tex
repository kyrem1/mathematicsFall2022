\documentclass[12pt,letterpaper]{article}

%--------Packages--------
\usepackage{amsmath, amsthm, amssymb}
\usepackage{xspace}
\usepackage{graphicx}
\usepackage{hhline}
\usepackage{amssymb}
\usepackage{array}
\usepackage{braket}
\usepackage{multicol}
\usepackage{mathtools}
\usepackage{enumerate}
\usepackage{delarray}
\usepackage{mathtools}
\usepackage{fullpage}
\usepackage{faktor} % For quotients
\usepackage{mathrsfs}

\usepackage[italicdiff]{physics} % For differentials
\usepackage{bbm} % For indicator

% \usepackage{quiver}
\usepackage[linguistics]{forest}




%--------Page Setup--------

\pagestyle{empty}%

\setlength{\hoffset}{-1.54cm}
\setlength{\voffset}{-1.54cm}

\setlength{\topmargin}{0pt}
\setlength{\headsep}{0pt}
\setlength{\headheight}{0pt}

\setlength{\oddsidemargin}{0pt}

\setlength{\textwidth}{195mm}
\setlength{\textheight}{250mm}


%--------Macros--------

\newcommand{\sub}{\subseteq}
\newcommand{\lcm}{\text{lcm}}
\newcommand{\mc}[1]{\mathcal{#1}}
\newcommand{\mf}[1]{\mathfrak{#1}}
\newcommand{\ms}[1]{\mathscr{#1}}
\newcommand{\sO}{\mathcal{O}}
\newcommand{\cyclic}[1]{\langle#1\rangle}
\newcommand{\units}[1]{#1 ^{\times}}
\newcommand{\la}{\langle}
\newcommand{\ra}{\rangle}
\newcommand{\lr}[1]{\left(#1\right)}
\newcommand{\lrvert}[1]{\left\lvert#1\right\rvert}

\DeclarePairedDelimiterX{\inp}[2]{\langle}{\rangle}{#1, #2}

%----Switch phi and varphi
% \let\temp\phi
% \let\phi\varphi
% \let\varphi\temp

\newcommand{\C}{\mathbb{C}}
\newcommand{\F}{\mathbb{F}}
\newcommand{\E}{\mathbb{E}}
\newcommand{\N}{\mathbb{N}\xspace}
\newcommand{\I}{\mathbb{I}\xspace}
\newcommand{\R}{\mathbb{R}\xspace}
\newcommand{\Z}{\mathbb{Z}\xspace}
\newcommand{\Q}{\mathbb{Q}\xspace}
\newcommand{\G}{\mathbb{G}\xspace}

\renewcommand{\H}{\mathcal{H}}
\newcommand{\M}{\mathcal{M}}

\DeclareMathOperator{\Spec}{Spec}
\DeclareMathOperator{\res}{res}
% \DeclareMathOperator{\Tr}{Tr}
\DeclareMathOperator{\ord}{ord}
\DeclareMathOperator{\Sym}{Sym}
% \DeclareMathOperator{\dv}{div}
\DeclareMathOperator{\alb}{alb}
\DeclareMathOperator{\img}{Im}
\DeclareMathOperator{\et}{et}
\DeclareMathOperator{\ck}{coker}
\DeclareMathOperator{\Reg}{Reg}
\DeclareMathOperator{\Cor}{Cor}
\DeclareMathOperator{\Ac}{at}
\DeclareMathOperator{\supp}{supp}
\DeclareMathOperator{\Hom}{Hom}
\DeclareMathOperator{\Pic}{Pic}
\DeclareMathOperator{\Gal}{Gal}
\DeclareMathOperator{\fc}{frac}
\DeclareMathOperator{\Ann}{Ann}
\DeclareMathOperator{\Mod}{Mod}
\DeclareMathOperator{\Cone}{Cone}
\DeclareMathOperator{\FI}{FI}
\DeclareMathOperator{\End}{End}
\DeclareMathOperator{\Alb}{Alb}
\DeclareMathOperator{\Ext}{Ext}
\DeclareMathOperator{\ab}{ab}
\DeclareMathOperator{\Jac}{Jac}
\DeclareMathOperator{\coker}{coker}
\DeclareMathOperator{\fr}{frac}
\DeclareMathOperator{\Int}{Int}
\let\Span\relax
\DeclareMathOperator{\Span}{Span}
\DeclareMathOperator{\Ran}{Ran}



%----Analysis
\newcommand{\summ}{\sum\limits}
% \newcommand{\norm}[1]{\left\lVert#1\right\rVert}
\newcommand{\thicc}{\bigg}
\newcommand{\eps}{\varepsilon}
\newcommand*\cls[1]{\overline{#1}}
\newcommand{\ind}{\mathbbm{1}}
\DeclareMathOperator{\sgn}{sgn}


%--------Theorem environments--------
\newtheorem{definition}{Definition}[]
\newtheorem{lemma}{Lemma}[]
\newtheorem{corollary}{Corollary}[]
\newtheorem{theorem}{Theorem}[]
\theoremstyle{remark}
\newtheorem*{claim}{Claim}


\newenvironment{solution}
{\begin{proof}[Solution]}
{\end{proof}}


\makeatletter
\newcommand{\thickhline}{%
    \noalign {\ifnum 0=`}\fi \hrule height 1pt
    \futurelet \reserved@a \@xhline
}
\newcolumntype{"}{@{\hskip\tabcolsep\vrule width 1pt\hskip\tabcolsep}}
\makeatother

% --------Problem environment--------
\setlength\parindent{0pt}
\setcounter{secnumdepth}{0}
\newcounter{partCounter}
\newcounter{homeworkProblemCounter}
\setcounter{homeworkProblemCounter}{1}


\newenvironment{homeworkProblem}[1][-1]{
    \ifnum#1>0
        \setcounter{homeworkProblemCounter}{#1}
    \fi
    \section{Problem \arabic{homeworkProblemCounter}}
    \setcounter{partCounter}{1}
    \stepcounter{homeworkProblemCounter}
}


%--------Metadata--------
\title{MATH 7410 Homework 1}
\author{James Harbour}

\begin{document}
\maketitle

\begin{homeworkProblem}
    Let $ L:\R^n\to\R^m $ be a linear map. For any $ p\in\R^n $, there is a canonical identification $ T_p(\R^n)\to \R^n $ given by 
    \[
        \sum a^i \pdv{x^i}\vert_p\mapsto a = (a^1,\ldots, a^n)
    \]

    Show that the differential $ L_{*,p}:T_p(R^n)\to T_{L(p)}(R^m) $ is the map $ L:\R^n\to \R^m $ itself, with the identification of the tangent spaces as above.



\end{homeworkProblem}

    

\begin{homeworkProblem}
   If $ M $ and $ N  $ are manifolds, let $ \pi_1:M\times N\to M $ and $ \pi_2:M\times N \to N $ be the two projections. Prove that for $ (p,q)\in M\times N  $, 
   \[
       (\pi_{1*},\pi_{2*}: T_{(p,q)}(M\times N) \to T_p M \times T_q N
   \]
   is an isomorphism.




\end{homeworkProblem}


\begin{homeworkProblem}
    Let $ G $ be a Lie group with multiplication map $ \mu: G\times G\to G $, 
    inverse map $ \iota:G\to G $, and identity element $ e $. \\

    \textbf{(a)}: Show that the differential at the identity of the multiplication
    map $ \mu $ is addition: 
    \begin{align*}
        \mu_{*,(e,e)}:T_e G \times T_e G\to T_e G, \\
        \mu_{*,(e,e)}(X_e,Y_e) = X_e + Y_e.
    \end{align*} 
    (\emph{Hint}: First, compute $ \mu_{*,(e,e)}(X_e,0) $ and $ \mu_{*,(e,e)}(0,Y_e) $ using Proposition 8.18). \\
    
    \begin{proof}
        Let $ X_e, Y_e\in T_e G $ and choose curves $\alpha_1,\alpha_2:
        (-\epsilon,\epsilon)\to G  $ such that $\alpha_{1}(0)=\alpha_{2}(0)= 
        e $ and $\dv{\alpha_1}{t}\vert_0=X_e $, $\dv{\alpha_{2}}{t} \vert_0= Y_e$.
        Consider $ \beta_{1},\beta_{2} $ given by $ \beta_1(t) = (\alpha_{1}(t),e) $ and $ \beta_{2}(t) = (e,\alpha_{2}(t)) $. Then  
    \end{proof}

    \textbf{(b)}: Show that the differential at the identity of $\iota$ is the negative:
    \begin{align*}
        \iota_{*,e}:T_e G\to T_e G\\ 
        \iota_{*,e}(X_e) = -X_e.
    \end{align*}

    (\emph{Hint}: Take the differential of $ \mu(c(t), (t\circ c)(t)) = e $.)
    
    
    


\end{homeworkProblem}

\begin{homeworkProblem}
 Show that $ T_p^1 M\sub T_p^2 M $.


\end{homeworkProblem}


\end{document}
