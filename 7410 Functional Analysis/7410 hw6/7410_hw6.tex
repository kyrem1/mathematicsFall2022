\documentclass[12pt,letterpaper]{article}

%--------Packages--------
\usepackage{amsmath, amsthm, amssymb}
\usepackage{xspace}
\usepackage{graphicx}
\usepackage{hhline}
\usepackage{amssymb}
\usepackage{array}
\usepackage{braket}
\usepackage{multicol}
\usepackage{mathtools}
\usepackage{enumerate}
\usepackage{delarray}
\usepackage{mathtools}
\usepackage{fullpage}
\usepackage{faktor} % For quotients
\usepackage{mathrsfs}

\usepackage[italicdiff]{physics} % For differentials
\usepackage{bbm} % For indicator

% \usepackage{quiver}
\usepackage[linguistics]{forest}




%--------Page Setup--------

\pagestyle{empty}%

\setlength{\hoffset}{-1.54cm}
\setlength{\voffset}{-1.54cm}

\setlength{\topmargin}{0pt}
\setlength{\headsep}{0pt}
\setlength{\headheight}{0pt}

\setlength{\oddsidemargin}{0pt}

\setlength{\textwidth}{195mm}
\setlength{\textheight}{250mm}


%--------Macros--------

\newcommand{\sub}{\subseteq}
\newcommand{\lcm}{\text{lcm}}
\newcommand{\mc}[1]{\mathcal{#1}}
\newcommand{\mf}[1]{\mathfrak{#1}}
\newcommand{\ms}[1]{\mathscr{#1}}
\newcommand{\sO}{\mathcal{O}}
\newcommand{\cyclic}[1]{\langle#1\rangle}
\newcommand{\units}[1]{#1 ^{\times}}
\newcommand{\la}{\langle}
\newcommand{\ra}{\rangle}
\newcommand{\lr}[1]{\left(#1\right)}
\newcommand{\lrvert}[1]{\left\lvert#1\right\rvert}

\DeclarePairedDelimiterX{\inp}[2]{\langle}{\rangle}{#1, #2}

%----Switch phi and varphi
% \let\temp\phi
% \let\phi\varphi
% \let\varphi\temp

\newcommand{\C}{\mathbb{C}}
\newcommand{\F}{\mathbb{F}}
\newcommand{\E}{\mathbb{E}}
\newcommand{\N}{\mathbb{N}\xspace}
\newcommand{\I}{\mathbb{I}\xspace}
\newcommand{\R}{\mathbb{R}\xspace}
\newcommand{\Z}{\mathbb{Z}\xspace}
\newcommand{\Q}{\mathbb{Q}\xspace}
\newcommand{\G}{\mathbb{G}\xspace}

\renewcommand{\H}{\mathcal{H}}
\newcommand{\M}{\mathcal{M}}

\DeclareMathOperator{\Spec}{Spec}
\DeclareMathOperator{\res}{res}
% \DeclareMathOperator{\Tr}{Tr}
\DeclareMathOperator{\ord}{ord}
\DeclareMathOperator{\Sym}{Sym}
% \DeclareMathOperator{\dv}{div}
\DeclareMathOperator{\alb}{alb}
\DeclareMathOperator{\img}{Im}
\DeclareMathOperator{\et}{et}
\DeclareMathOperator{\ck}{coker}
\DeclareMathOperator{\Reg}{Reg}
\DeclareMathOperator{\Cor}{Cor}
\DeclareMathOperator{\Ac}{at}
\DeclareMathOperator{\supp}{supp}
\DeclareMathOperator{\Hom}{Hom}
\DeclareMathOperator{\Pic}{Pic}
\DeclareMathOperator{\Gal}{Gal}
\DeclareMathOperator{\fc}{frac}
\DeclareMathOperator{\Ann}{Ann}
\DeclareMathOperator{\Mod}{Mod}
\DeclareMathOperator{\Cone}{Cone}
\DeclareMathOperator{\FI}{FI}
\DeclareMathOperator{\End}{End}
\DeclareMathOperator{\Alb}{Alb}
\DeclareMathOperator{\Ext}{Ext}
\DeclareMathOperator{\ab}{ab}
\DeclareMathOperator{\Jac}{Jac}
\DeclareMathOperator{\coker}{coker}
\DeclareMathOperator{\fr}{frac}
\DeclareMathOperator{\Int}{Int}
\let\Span\relax
\DeclareMathOperator{\Span}{Span}
\DeclareMathOperator{\Ran}{Ran}
\DeclareMathOperator{\ran}{ran}
\DeclareMathOperator{\ext}{ext}
\DeclareMathOperator{\Prob}{Prob}


%----Analysis
\newcommand{\summ}{\sum\limits}
% \newcommand{\norm}[1]{\left\lVert#1\right\rVert}
\newcommand{\thicc}{\bigg}
\newcommand{\eps}{\varepsilon}
\newcommand*\cls[1]{\overline{#1}}
\newcommand{\ind}{\mathbbm{1}}
\DeclareMathOperator{\sgn}{sgn}


%--------Theorem environments--------
\newtheorem{definition}{Definition}[]
\newtheorem{lemma}{Lemma}[]
\newtheorem{corollary}{Corollary}[]
\newtheorem{theorem}{Theorem}[]
\theoremstyle{remark}
\newtheorem*{claim}{Claim}


\newenvironment{solution}
{\begin{proof}[Solution]}
{\end{proof}}


\makeatletter
\newcommand{\thickhline}{%
    \noalign {\ifnum 0=`}\fi \hrule height 1pt
    \futurelet \reserved@a \@xhline
}
\newcolumntype{"}{@{\hskip\tabcolsep\vrule width 1pt\hskip\tabcolsep}}
\makeatother

% --------Problem environment--------
\setlength\parindent{0pt}
\setcounter{secnumdepth}{0}
\newcounter{partCounter}
\newcounter{homeworkProblemCounter}
\setcounter{homeworkProblemCounter}{1}


\newenvironment{homeworkProblem}[1][-1]{
    \ifnum#1>0
        \setcounter{homeworkProblemCounter}{#1}
    \fi
    \section{Problem \arabic{homeworkProblemCounter}}
    \setcounter{partCounter}{1}
    \stepcounter{homeworkProblemCounter}
}


%--------Metadata--------
\title{MATH 7410 Homework 6}
\author{James Harbour}

\begin{document}
\maketitle

\begin{homeworkProblem}
    Let $ G $ be a finitely generated group with finite generation set $ S $. Suppose that $ S $ is symmetric and contains the identity. We let 
    \[
        B_{S}(n)= \{s_{1}\cdots s_{n}: s_{i}\in S   , i=1,\ldots, n\}.
    \]
    Suppose that $ G $ has \textit{subexponential growth}, namely $ \limsup_{n\to\infty}|B_{S}(n)|^{1/n}=1 $ (note that this implies that the limit itself is $ 1 $). Show that there is a subsequence $ n_{1}<n_{2}<\cdots $ of natural numbers so that $ (B_{S}(n_{k}))_{k=1}^{\infty} $ is a Folner sequence.
    \[
        \liminf_{n\to\infty}\frac{a_{n}}{a_{n-k}} \leq \liminf_{n\to\infty}a_{n}^{k/n}.
    \]
    
    \begin{proof}
        For $ g\in G $, let $ l_{S}(g) $ be the reduced length of $ g $ when written as an $ S $-word omitting occurrences of the identity and set $ l_{S}(e) = 0 $. Since $ S $ is fixed, for brevity we write $ B(n) = B_{S}(n) $. Note that, for $ g\in G $ we have that $ gB(n),B(n)\sub B(n+l_{S}(g)) $, so 
        \begin{align*}
            \frac{|g B(n)\,\Delta\, B(n)|}{|B(n)|} &= \frac{|g B(n) \setminus B(n)|}{|B(n)|} + \frac{|B(n)\setminus gB(n)|}{|B(n)|}\\
            &\leq \frac{|B(n+l_{S}(g))\setminus B(n)|}{|B(n)|} + \frac{|B(n+l_{S}(g)) \setminus gB(n)|}{|B(n)|} \\
            &\leq  2\frac{|B(n+l_{S}(g))\setminus B(n)|}{|B(n)|} = 2\frac{|B(n+l_{S}(g))|}{|B(n)|} - 2.
        \end{align*}
        Note that, for $ k\in \N $, 
        \[
            \liminf_{n\to\infty} \frac{|B(n+k)|}{|B(n)|} \leq \liminf_{n\to\infty}|B(n+k)|^{\frac{k}{n+k}} = 1.
        \]
        Choose a subsequence $ (n_{k})_{k=1}^{\infty} $ as follows: choose $ n_{1} $ such that $ \frac{|B(n_{2}+1)|}{|B(n_{1})|}\leq 1+\frac{1}{1} $. Having chosen $ n_{1}<\ldots< n_{k-1} $, choose $ n_{k}>n_{k-1} $ such that $ \frac{|B(n_{k}+k)|}{|B(n_{k})|}\leq 1+\frac{1}{k} $. Then
        \[
            \limsup_{k\to\infty}\frac{|B(n_{k}+k)|}{|B(n_{k})|} \leq 1.
        \]
        
        Hence, for $ g\in G $,
        \begin{align*}
            \limsup_{k\to\infty}\frac{|gB(n_{k})\Delta B(n_{k})|}{|B(n_{k})|} \leq 2\limsup_{k\to\infty} \frac{|B(n_{k}+l_{S}(g))|}{|B(n_{k})|}-2 &\leq 2\limsup_{k\to\infty} \frac{|B(n_{k}+k)|}{|B(n_{k})|}-2 \leq 0
        \end{align*}
    \end{proof}
    
    
\end{homeworkProblem}



\begin{homeworkProblem}
    Let $ G $ be a countable, discrete group. For $ p\in [1,\infty) $ we say $ (f_{n})_{n=1}^{\infty} $ in $ l^{p}(G) $ are almost invariant vectors if $ \norm{f_{n}}_{p} = 1 $ and if 
    \[
        \norm{\lambda_{g}f_{n}-f_{n}}_{p}\xrightarrow{n\to\infty}0  \text{ for all } g\in G.
    \]
    
    \textbf{(a)}: For $ p\in [1,+\infty) $ and $ f\in l^{p}(G) $ prove that $ \norm{\lambda_{g}|f|-|f|}_{p}\leq \norm{\lambda_{g}f-f}_{p} $ for all $ g\in G $.

    \begin{proof}
        By the reverse triangle inequality, we have that $ |\lambda_{g}|f|-|f||\leq | \lambda_{g}f-f| $ pointwise. Now,
        \[
            \norm{\lambda_{g}|f|-|f|}_{p}^{p} = \int |\lambda_{g}|f|-|f||^{p}\dd{\mu}\leq\int | \lambda_{g}f-f|^{p}\dd{\mu}\leq  \norm{\lambda_{g}f-f}_{p}^{p},
        \]
        whence the result follows.
    \end{proof}

    \textbf{(b)}: For $ a,b\in [0,+\infty) $ and $ p\in [1,+\infty) $ prove that $ |a^{1/p}-b^{1/p}|\leq |a-b|^{1/p} $ and 
    \[
        |a^{p}-b^{p}|\leq p|a-b|\max(a^{p-1},b^{p-1}) \leq p|a-b| (a^{p-1}+ b^{p-1}).
    \]

    \begin{proof}
        The first inequality follows from homework 1 problem 1 part (a). 
        Note that the second inequality is trivial if $ a $ or $ b $ is zero or if $ p=1 $, so assume $ a,b>0 $ and $ p>1 $. \\

        Consider the polynomial $ f(x) = x^{p}+p(1-x)-1 $ on the interval $ (0,1] $. Computing $ f '(x) = px^{p-1}-p $, the only critical points for $ f $ are at $ x=1 $ whence $ f(x) = 0 $. As $ f ' < 0 $ for all $ x\in (0,1) $ and $ f(0) = p-1 > 0$, it follows that $ f(x) \geq f(1) = 0 $ for all $ x\in (0,1] $.\\

       Without loss of generality, assume $ b\leq a $. Consider $ x = \frac{b}{a}\leq 1 $. By the nonegativity of the above polynomial, 
       \[
           1-\frac{b^{p}}{a^{p}} = 1-x^{p} \leq p(1-x) = p \frac{(a-b)a^{p-1}}{a}
       \]
       whence the second inequality follows.  
   \end{proof}

    \textbf{(c)}: Suppose $ p\in[1,+\infty) $. Prove that there are almost invariant vectors in $ l^{p}(G) $ if and only if $ G $ is amenable.

    \begin{proof}\ \\
        \underline{$ \implies $}: Suppose $ (f_{n})_{n=1}^{\infty} $ is a sequence of almost invariant unit vectors in $ l^{p}(G) $, and fix $ g\in G $. Let $ \mu_{n}:=|f_{n}|^{p} $ and note that $ \mu_{n}\in \Prob(G)\sub l^{1}(G) $. As $ \norm{f_{n}} = 1 $, $ |f_{n}|\geq 0 $, and $ G $ is discrete, it follows that $ |f_{n}|\leq \norm{f_{n}} = 1$. By part $ (a) $, it follows that 
        \[
            \norm{\lambda_{g}|f_{n}| - |f_{n}|}_{p} \leq \norm{\lambda_{g}f_{n}-f_{n}}_{p}\xrightarrow{n\to\infty}0.
        \]
        Now observe that, applying Holder's inequality with conjugate exponents $ p, \frac{p}{p-1} $,
        \begin{align*}
            \norm{\lambda_{g} \mu_{n} - \mu_{n}}_{1} = \int \lrvert{| \lambda_{g}f_{n}|^{p} - |f_{n}|^{p}}\dd{\mu} &\leq p\int \lrvert{| \lambda_{g}f_{n}| - |f_{n}|}\cdot \max\left\{|\lambda_{g}f_{n}|^{p-1}, |f_{n}|^{p-1}\right\}\dd{\mu} \\
            &\leq p \norm{\lambda_{g}|f_{n}| - |f_{n}|}_{p}\cdot \norm{\max\left\{|\lambda_{g}f_{n}|^{p-1}, |f_{n}|^{p-1}\right\}}_{\frac{p}{p-1}}  \\
            &\leq p \norm{\lambda_{g}|f_{n}| - |f_{n}|}_{p}\cdot \lr{\int \max\left\{|\lambda_{g}f_{n}|^{p-1}, |f_{n}|^{p-1}\right\}^{\frac{p}{p-1}}}^{\frac{p-1}{p}}\\
            &\leq p \norm{\lambda_{g}|f_{n}| - |f_{n}|}_{p}\cdot \lr{\int \max\left\{|\lambda_{g}f_{n}|^{p}, |f_{n}|^{p}\right\}}^{\frac{p-1}{p}}\\
            &\leq p \norm{\lambda_{g}|f_{n}| - |f_{n}|}_{p}\cdot \lr{\int |\lambda_{g}f_{n}|^{p} + |f_{n}|^{p}}^{\frac{p-1}{p}} \\
            &\leq2^{\frac{p-1}{p}} p \norm{\lambda_{g}|f_{n}| - |f_{n}|}_{p}\xrightarrow{n\to\infty}0
        \end{align*}

        \underline{$ \impliedby $}: Suppose that $ G $ is amenable and $ p\in [1,+\infty) $. Choose a sequence $ (\mu_{n})_{n=1}^{\infty} $ of almost invariant probability measures for $ G $. Set $ f_{n} = \mu_{n}^{1/p} $. Then $ f_{n}\in l^{p}(G) $ and $ \norm{f_{n}}_{p} = 1 $. So, we compute that
        \[
            \norm{\lambda_{g}f_{n} - f_{n}}_{p}^p = \int \lrvert{ \lambda_{g} \mu_{n}^{1/p}-\mu_{n}^{1/p}}^{p}\dd{\mu} \leq \int | \lambda_{g} \mu_{n}-\mu_{n}|\dd{\mu} \xrightarrow{n\to\infty}0
        \]
    \end{proof}
\end{homeworkProblem}





\end{document}
