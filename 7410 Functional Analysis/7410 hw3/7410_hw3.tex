\documentclass[12pt,letterpaper]{article}

%--------Packages--------
\usepackage{amsmath, amsthm, amssymb}
\usepackage{xspace}
\usepackage{graphicx}
\usepackage{hhline}
\usepackage{amssymb}
\usepackage{array}
\usepackage{braket}
\usepackage{multicol}
\usepackage{mathtools}
\usepackage{enumerate}
\usepackage{delarray}
\usepackage{mathtools}
\usepackage{fullpage}
\usepackage{faktor} % For quotients
\usepackage{mathrsfs}

\usepackage[italicdiff]{physics} % For differentials
\usepackage{bbm} % For indicator

% \usepackage{quiver}
\usepackage[linguistics]{forest}




%--------Page Setup--------

\pagestyle{empty}%

\setlength{\hoffset}{-1.54cm}
\setlength{\voffset}{-1.54cm}

\setlength{\topmargin}{0pt}
\setlength{\headsep}{0pt}
\setlength{\headheight}{0pt}

\setlength{\oddsidemargin}{0pt}

\setlength{\textwidth}{195mm}
\setlength{\textheight}{250mm}


%--------Macros--------

\newcommand{\sub}{\subseteq}
\newcommand{\lcm}{\text{lcm}}
\newcommand{\mc}[1]{\mathcal{#1}}
\newcommand{\mf}[1]{\mathfrak{#1}}
\newcommand{\ms}[1]{\mathscr{#1}}
\newcommand{\sO}{\mathcal{O}}
\newcommand{\cyclic}[1]{\langle#1\rangle}
\newcommand{\units}[1]{#1 ^{\times}}
\newcommand{\la}{\langle}
\newcommand{\ra}{\rangle}
\newcommand{\lr}[1]{\left(#1\right)}
\newcommand{\lrvert}[1]{\left\lvert#1\right\rvert}

\DeclarePairedDelimiterX{\inp}[2]{\langle}{\rangle}{#1, #2}

%----Switch phi and varphi
% \let\temp\phi
% \let\phi\varphi
% \let\varphi\temp

\newcommand{\C}{\mathbb{C}}
\newcommand{\F}{\mathbb{F}}
\newcommand{\E}{\mathbb{E}}
\newcommand{\N}{\mathbb{N}\xspace}
\newcommand{\I}{\mathbb{I}\xspace}
\newcommand{\R}{\mathbb{R}\xspace}
\newcommand{\Z}{\mathbb{Z}\xspace}
\newcommand{\Q}{\mathbb{Q}\xspace}
\newcommand{\G}{\mathbb{G}\xspace}

\renewcommand{\H}{\mathcal{H}}
\newcommand{\M}{\mathcal{M}}

\DeclareMathOperator{\Spec}{Spec}
\DeclareMathOperator{\res}{res}
% \DeclareMathOperator{\Tr}{Tr}
\DeclareMathOperator{\ord}{ord}
\DeclareMathOperator{\Sym}{Sym}
% \DeclareMathOperator{\dv}{div}
\DeclareMathOperator{\alb}{alb}
\DeclareMathOperator{\img}{Im}
\DeclareMathOperator{\et}{et}
\DeclareMathOperator{\ck}{coker}
\DeclareMathOperator{\Reg}{Reg}
\DeclareMathOperator{\Cor}{Cor}
\DeclareMathOperator{\Ac}{at}
\DeclareMathOperator{\supp}{supp}
\DeclareMathOperator{\Hom}{Hom}
\DeclareMathOperator{\Pic}{Pic}
\DeclareMathOperator{\Gal}{Gal}
\DeclareMathOperator{\fc}{frac}
\DeclareMathOperator{\Ann}{Ann}
\DeclareMathOperator{\Mod}{Mod}
\DeclareMathOperator{\Cone}{Cone}
\DeclareMathOperator{\FI}{FI}
\DeclareMathOperator{\End}{End}
\DeclareMathOperator{\Alb}{Alb}
\DeclareMathOperator{\Ext}{Ext}
\DeclareMathOperator{\ab}{ab}
\DeclareMathOperator{\Jac}{Jac}
\DeclareMathOperator{\coker}{coker}
\DeclareMathOperator{\fr}{frac}
\DeclareMathOperator{\Int}{Int}
\let\Span\relax
\DeclareMathOperator{\Span}{Span}
\DeclareMathOperator{\Ran}{Ran}
\DeclareMathOperator{\ran}{ran}


%----Analysis
\newcommand{\summ}{\sum\limits}
% \newcommand{\norm}[1]{\left\lVert#1\right\rVert}
\newcommand{\thicc}{\bigg}
\newcommand{\eps}{\varepsilon}
\newcommand*\cls[1]{\overline{#1}}
\newcommand{\ind}{\mathbbm{1}}
\DeclareMathOperator{\sgn}{sgn}


%--------Theorem environments--------
\newtheorem{definition}{Definition}[]
\newtheorem{lemma}{Lemma}[]
\newtheorem{corollary}{Corollary}[]
\newtheorem{theorem}{Theorem}[]
\theoremstyle{remark}
\newtheorem*{claim}{Claim}


\newenvironment{solution}
{\begin{proof}[Solution]}
{\end{proof}}


\makeatletter
\newcommand{\thickhline}{%
    \noalign {\ifnum 0=`}\fi \hrule height 1pt
    \futurelet \reserved@a \@xhline
}
\newcolumntype{"}{@{\hskip\tabcolsep\vrule width 1pt\hskip\tabcolsep}}
\makeatother

% --------Problem environment--------
\setlength\parindent{0pt}
\setcounter{secnumdepth}{0}
\newcounter{partCounter}
\newcounter{homeworkProblemCounter}
\setcounter{homeworkProblemCounter}{1}


\newenvironment{homeworkProblem}[1][-1]{
    \ifnum#1>0
        \setcounter{homeworkProblemCounter}{#1}
    \fi
    \section{Problem \arabic{homeworkProblemCounter}}
    \setcounter{partCounter}{1}
    \stepcounter{homeworkProblemCounter}
}


%--------Metadata--------
\title{MATH 7410 Homework 3 (In-Progress)}
\author{James Harbour}

\begin{document}
\maketitle

\begin{homeworkProblem}
    \textbf{(a)}: Let $ X $ be a separable Banach space. Show that $ Ball(X^{*})=\{\phi\in X^{*}:\norm{\phi}\leq1\} $ is $ wk^{*} $-metrizable. 
    
    \begin{proof}
        Note that it suffices to show that a countable subset of the seminorms defining the LCS topology on $ X^{*} $ in fact define the topology on $ Ball(X^{*}) $.\\

        Choose a norm dense sequence $ (x_{n})_{n=1}^{\infty} $ in $ X $. We claim that the seminorms $ \rho_{x_{n}} = |ev_{x_{n}}(\cdot)| : X^{*}\to[0,+\infty) $ define the restriction of the $ wk^{*} $-topology to $ Ball(X^{*}) $. \\

        Suppose that $ (\phi_{\alpha})_{\alpha\in I} $ is a net in $ Ball(X^{*}) $and $ \phi\in Ball(X^{*}) $ is such that $ \phi_{\alpha}(x_{n})\to \phi(x) $ for all $ n\in\N $. Now let $ x\in X $ and fix $ \eps>0 $. Then by density there is some  $ n\in \N $ such that $ \norm{x-x_{n}}< \eps/3 $. Moreover, by assumption, there is some $ \alpha_{0}\in I $ such that for all $ \alpha \geq \alpha_{0} $, $ | \phi_{\alpha}(x_{n})-\phi(x_{n}) | <\eps/3 $. Now, for all $ \alpha\geq \alpha_{0} $, 
        \begin{align*}
            | \phi(x) - \phi_{\alpha}(x)| \leq | \phi(x-x_{n}) | + | \phi(x_{n}) - \phi_{\alpha}(x)| &\leq \norm{x-x_{n}}+| \phi_{\alpha}(x_{n}) -\phi(x_{n}) | + | \phi_{\alpha}(x_{n}) - \phi_{\alpha}(x)| \\
            &<\frac{\eps}{3}+\frac{\eps}{3}+ \norm{\phi_{\alpha}}\norm{x_{n}-x} <\eps
        \end{align*}
        so in fact, for nets in $ Ball(X^{*}) $, pointwise convergence on $ (x_{n})_{n}^{\infty} $ implies pointwise convergence everywhere, and thus $ wk^{*} $-convergence of the underlying nets. 
    \end{proof}

    \textbf{(b)}: If $ X $ is a Banach space, show that there is a compact space $ K $ such that $ X $ is isometrically isomorphic to a closed subspace of $ C(K) $. 

\end{homeworkProblem}


\begin{homeworkProblem}
    Let $ Bil(X\times Y, Z) $ be the space of bounded, bilinear maps from $ X\times Y\to Z $. \\

    \textbf{(a)}: Suppose that $ B_{x},B^{y} $ are bounded for each $ x\in X $,$ y\in Y $. Prove that there is a constant $ M>0 $ so that 
    \[
        \norm{B(x,y)}\leq M \norm{x}\norm{y}
    \]
    (use the Principle of Uniform Boundedness).

    \begin{proof}
        For $ x\in X $, $ y\in Ball(Y) $, note that $ \norm{B_{y}(x)}\leq \norm{B_{x}} $.
        Then by the principle of uniform boundedness, $ C:=\sup_{y\in Ball(Y)}\norm{B_{y}}<+\infty $. Now observe that
        \[
            \sup_{x\in Ball(X),\, y\in Ball(Y)} \norm{B(x,y)} = \sup_{x\in Ball(X),\, y\in Ball(Y)} \norm{B_{y}(x)} \leq \sup_{y\in Ball(Y)} \norm{B_{y}} = C,
        \]
        whence the claim follows by scaling.
    \end{proof}

    \textbf{(b)}: Show that the map $ \Phi:Bil(X\times Y,\F)\to B(X,Y^{*}) $ given by $ [\widetilde{\Phi}(B)(x)](y)=B(x,y) $ is a well-defined, isometric isomorphism. \\

    \textbf{(c)}: By switching names, it follows that the map $ \widetilde{\Phi}: Bil(X\times Y,\F)\to B(Y,X^{*}) $ given by $ [\widetilde{\Phi}(B)(y)](x) = B(x,y) $ is a well-defined, isometric isomorphism. So the map $ \widetilde{\Phi}\circ \Phi^{-1} $ is an isometric isomorphism $ B(X,Y^{*})\cong B(Y,X^{*}) $. What is this isomorphism?




\end{homeworkProblem}


\begin{homeworkProblem}
    Let $ X,Y $ be Banach spaces. And let $ (T_{n})_{n=1}^{\infty} $ be a sequence in $ B(X,Y) $. 
    
    \begin{lemma}
        If $ X $ is a Banach space, $ (x_{n})_{n=1}^{\infty} $ a sequence in $ X $ such that $ \phi(x_{n}) $ is a bounded sequence for all $ \phi\in X^{*} $, then $ (x_{n})_{n=1}^{\infty} $ is bounded in norm.
    \end{lemma}
  
    \begin{proof}[Proof of Lemma.]
        Let $ \hat{x}\in X^{**} $ denote the canonical image of $ x\in X $ inside $ X^{**} $. For each $ \phi\in X^{*} $, $ \sup_{n\in\N} |\hat{x}_{n}(\phi)| = \sup_{n\in\N}| \phi(x_{n})| < +\infty $, so by the principle of uniform boundedness, $ \sup_{n\in\N}\norm{x_{n}}=\sup_{n\in\N}\norm{\hat{x}_{n}} <+\infty $.
    \end{proof}

    \textbf{(a)}: If $ T_{n} $ converges in the WOT to $ T\in B(X,Y) $ show that $ \sup_{n}\norm{T_{n}}<+\infty $. (In particular, if $ T_{n} $ converges strongly, then it is norm).

    \begin{proof}
        Fix $ x\in X $. Then for all $ \phi\in Y^{*} $, $ \phi(T_{n}x)\to \phi(Tx) $ so $ \sup_{n\in\N}| \phi(T_{n}x)|<+\infty $. Now by the above lemma, $ \sup_{n\in\N}\norm{T_{n}x}<+\infty $. Hence, by the principle of uniform boundedness, $ \sup_{n\in\N}\norm{T_{n}}<+\infty $ as desired.
    \end{proof}

    \textbf{(b)}: If $ \sup_{n}\norm{T_{n}}<+\infty $ and there is a norm dense $ D\sub X $ so that $ T_{n}x $ converges for every $ x\in D$, show that $ T_{n}x $ converges for all $ x\in X $, that $ Tx = \lim_{n\to\infty}T_{n}x $ is a bounded operator, and that $ \norm{Tx-T_{n}x}\xrightarrow{n\to\infty}0 $ for every $ x\in X $. 


\end{homeworkProblem}


\begin{homeworkProblem}
    Let $ X,Y $ be Banach spaces. And let $ (T_{n})_{n=1}^{\infty} $ be a sequence in $ B(X,Y) $. Suppose that $ \sup_{n}\norm{T_{n}} <+\infty $ and that $ D\sub X $, $ G\sub Y^{*} $ are norm dense. Assume that $ \lim_{n}\phi(T_{n}x) $ exists for all $ \phi\in G $,$ x\in D $. \\
    
    \textbf{(a)}: Show that $ \lim_{n} \phi(T_{n}x) $ exists for all $ \phi\in Y^{*} $, $ x\in X $. 

    \begin{proof}
        Fix $ x\in X, \phi\in Y^{*} $, and $ \eps>0 $. Choose $ y\in D, \psi\in G $ such that $ \norm{x-y},\norm{\phi-\psi}<\eps $. Set $ C = \sup_{n\in\N}\norm{T_{n}} $ Then, for $ n,m\in\N $, we compute that 
        \begin{align*}
            \norm{\phi(T_{n}x)-\phi(T_{m}x)} &\leq \norm{\phi(T_{n}x)-\psi(T_{n}x)}+\norm{\psi(T_{n}x)-\phi(T_{m}x)} \\
            &\leq C\eps \norm{x} + \norm{\psi(T_{n}x)-\psi(T_{n}y)} + \norm{\psi(T_{n}y)-\phi(T_{m}x)} \\
            &\leq C\eps (\norm{x}+\norm{\psi}) + \norm{\psi(T_{n}y)-\psi(T_{m}x)} + \norm{\psi(T_{m}x)-\phi(T_{m}x)}\\
            &\leq C\eps (2\norm{x}+\norm{\psi}) + \norm{\psi(T_{n}y) - \psi(T_{m}y)} + \norm{\psi(T_{m}y)-\psi(T_{m}x)}\\
            &\leq C\eps (2\norm{x}+2\norm{\psi}) + \norm{\psi(T_{n}y) - \psi(T_{m}y)} \xrightarrow{n,m\to\infty,\,\eps\to0}0.
        \end{align*}
        Hence, by completeness of $ \F $, the desired limit exists.
    \end{proof}


    \textbf{(b)}: Show that for every $ x\in X $, there is a well-defined bounded operator $ S:X\to Y^{**} $ given by $ S(x)(\phi) = \lim_{n\to\infty} \phi(T_{n}x) $.  

    \begin{proof}
        This limit exists for every $ x\in X $ and $ \phi\in Y^{*} $ by part (a), and it is clearly linear in both $ x $ and $ \phi $. Now suppose that $ \phi\in Y^{**} $. Then
        \[ 
            |\lim_{n\to\infty} \phi(T_{n}x)| = \lim_{n\to\infty} | \phi(T_{n}x) | \leq \liminf_{n\to\infty}\norm{\phi}\norm{T_{n}x} \leq \liminf_{n\to\infty}\norm{\phi}\norm{T_{n}}\norm{x} \leq \norm{\phi}\norm{x}\sup_{n\in\N}\norm{T_{n}},
        \] 
        so $ \norm{S(x)}\leq \norm{x}\sup_{n\in\N}\norm{T_{n}}<+\infty $, whence $ S(x)\in Y^{**} $. Thus $ S $ is a well-defined operator from $ X $ to $ Y^{**} $. Moreover, the above inequality implies that $ \norm{S}\leq \sup_{n\in\N}\norm{T_{n}}<+\infty $, so $ S\in B(X,Y^{**}) $. 
    \end{proof}

    \textbf{(c)}: If $ T_{n}x $ converges weakly to an element of $ Y $ for every $ x\in D $, show that $ S(X)\sub Y $, and that $ T_{n}\to S $ WOT.

    \begin{proof}
        By assumption, for all $ x\in D $ there exists some $ y_{x}\in Y $ such that $ T_{n}x\to y_{x} $ weakly. 
    \end{proof}


\end{homeworkProblem}


\begin{homeworkProblem}
    Let $ G $ be a countable, discrete, group and $ \lambda:G\to B(l^{2}(G)) $ be given by $ (\lambda(g) \xi)(h) = \xi(g^{-1}h) $. \\

    \textbf{(a)}: Let $ (g_{n})_{n=1}^{\infty} $ be a sequence in $ G $ so that for every finite $ F\sub G $ we have $ \{n:g_{n}\in F\} $ is finite. Show that $ \lim_{n\to\infty} \lambda(g_{n})=0 $ in WOT. (Hint: consider first acting on pairs of vectors which are finitely supported and applying the preceding problem to reduce to this case).

    \begin{proof}
        Suppose first that $ \xi,\eta\in l^{2}(G) $ both have finite support, and let $ \supp(\xi) = \{x_{1},\ldots,x_{k}\} $, $ \supp(\eta) = \{y_{1},\ldots,y_{l}\} $, $ \alpha_{i} = \xi(x_{i}) $, $ \beta_{j}=\eta(y_{j}) $. Then, using finite supportedness to justify interchanges of summations, we compute that
        \[
            \inp{\lambda(g_{n}) \xi}{\eta} = \sum_{x\in G}\xi(g_{n}^{-1}x) \bar{\xi(x)} = \sum_{x\in G}\sum_{i,j=1}^{k,l} \alpha_{i} \bar{\beta_{j}} \delta_{x_{i}}(g_{n}^{-1}x) \delta_{y_{j}}(x) = \sum_{i,j=1}^{k,l} \sum_{x\in G}\alpha_{i} \bar{\beta_{j}} \delta_{x_{i}}(g_{n}^{-1}x) \delta_{y_{j}}(x) = \sum_{i,j=1}^{k,l} \alpha_{i} \bar{\beta_{j}} \delta_{g_{n}}(y_{j}x_{i}^{-1}).
        \]
        If $ g_{n}\not\in \bigcup_{j=1}^{l}\bigcup_{i=1}^{k}\{y_{j}x_{i}^{-1}\} $, then the above expression is zero. As this set is finite, the assumption on the given sequence implies that $ \inp{\lambda(g_{n}) \xi}{\eta} $ is eventually equal to zero past some fixed index, whence it converges to zero.  
            
        
    \end{proof}

    \textbf{(b)}: Suppose $ G $ is infinite. If $ \mc{K}\sub l^{2}(G) $ is closed and $ \lambda(g)\mc{K} = \mc{K} $ for every $ g\in G $, and $ \mc{K}\neq 0 $, show that $ \mc{K} $ is not finite-dimensional. (Hint: construct a sequence satisfies the hypothesis of the preceding problem. If $ \mc{K} $ is finite-dimensional, then $ \lambda $ applied to the sequence restricted to $ \mc{K} $ converges to $ 0 $ in WOT, and hence in any other LCS topology on $ B(\mc{K}) $. Consider using this for one of the other operator topologies to get a contradiction). 

    \begin{proof}
        Suppose, for the sake of contradiction, that $ \mc{K} $ is finite dimensional. Let $ (g_{n})_{n=1}^{\infty} $ be a sequence of pairwise distinct elements of $ G $. This sequence satisfies the hypothesis of part (a), whence $ \lambda(g_{n})\xrightarrow{\text{WOT}}0 $. As $ \mc{K} $ is $ \lambda(G) $-invariant, we have that $ \lambda(g_{n})\vert_{\mc{K}} \in B(\mc{K}) $ whence $ \lambda(g_{n})\vert_{\mc{K}}\xrightarrow{\text{WOT}}0 $ in $ B(\mc{K}) $. \\

        As $ \mc{K} $ is finite dimensional, $ B(\mc{K}) $ is also a finite dimensional LCS. Thus, every locally convex topology on $ B(K) $ is equal, whence $ \lambda(g_{n})\vert_{\mc{K}}\xrightarrow{\text{SOT}}0 $. Let $ \xi\in B(\mc{K}) $ with $ \xi\neq 0 $. Then 
        \[
            \norm{\xi} = \norm{\lambda(g_{n}) \xi} = \norm{\lambda(g_{n})\vert_{\mc{K}} \xi}\xrightarrow{n\to\infty}0
        \]
        which implies that $ \xi=0 $, contradicting the choice of $ \xi $.
    \end{proof}
\end{homeworkProblem}

\end{document}
