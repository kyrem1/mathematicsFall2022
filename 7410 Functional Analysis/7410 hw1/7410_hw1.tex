\documentclass[12pt,letterpaper]{article}

%--------Packages--------
\usepackage{amsmath, amsthm, amssymb}
\usepackage{xspace}
\usepackage{graphicx}
\usepackage{hhline}
\usepackage{amssymb}
\usepackage{array}
\usepackage{braket}
\usepackage{multicol}
\usepackage{mathtools}
\usepackage{enumerate}
\usepackage{delarray}
\usepackage{mathtools}
\usepackage{fullpage}
\usepackage{faktor} % For quotients
\usepackage{mathrsfs}

\usepackage[italicdiff]{physics} % For differentials
\usepackage{bbm} % For indicator

% \usepackage{quiver}
\usepackage[linguistics]{forest}




%--------Page Setup--------

\pagestyle{empty}%

\setlength{\hoffset}{-1.54cm}
\setlength{\voffset}{-1.54cm}

\setlength{\topmargin}{0pt}
\setlength{\headsep}{0pt}
\setlength{\headheight}{0pt}

\setlength{\oddsidemargin}{0pt}

\setlength{\textwidth}{195mm}
\setlength{\textheight}{250mm}


%--------Macros--------

\newcommand{\sub}{\subseteq}
\newcommand{\lcm}{\text{lcm}}
\newcommand{\mc}[1]{\mathcal{#1}}
\newcommand{\mf}[1]{\mathfrak{#1}}
\newcommand{\ms}[1]{\mathscr{#1}}
\newcommand{\sO}{\mathcal{O}}
\newcommand{\cyclic}[1]{\langle#1\rangle}
\newcommand{\units}[1]{#1 ^{\times}}
\newcommand{\la}{\langle}
\newcommand{\ra}{\rangle}
\newcommand{\lr}[1]{\left(#1\right)}
\newcommand{\lrvert}[1]{\left\lvert#1\right\rvert}

\DeclarePairedDelimiterX{\inp}[2]{\langle}{\rangle}{#1, #2}

%----Switch phi and varphi
% \let\temp\phi
% \let\phi\varphi
% \let\varphi\temp

\newcommand{\C}{\mathbb{C}}
\newcommand{\F}{\mathbb{F}}
\newcommand{\E}{\mathbb{E}}
\newcommand{\N}{\mathbb{N}\xspace}
\newcommand{\I}{\mathbb{I}\xspace}
\newcommand{\R}{\mathbb{R}\xspace}
\newcommand{\Z}{\mathbb{Z}\xspace}
\newcommand{\Q}{\mathbb{Q}\xspace}
\newcommand{\G}{\mathbb{G}\xspace}

\renewcommand{\H}{\mathcal{H}}
\newcommand{\M}{\mathcal{M}}

\DeclareMathOperator{\Spec}{Spec}
\DeclareMathOperator{\res}{res}
% \DeclareMathOperator{\Tr}{Tr}
\DeclareMathOperator{\ord}{ord}
\DeclareMathOperator{\Sym}{Sym}
% \DeclareMathOperator{\dv}{div}
\DeclareMathOperator{\alb}{alb}
\DeclareMathOperator{\img}{Im}
\DeclareMathOperator{\et}{et}
\DeclareMathOperator{\ck}{coker}
\DeclareMathOperator{\Reg}{Reg}
\DeclareMathOperator{\Cor}{Cor}
\DeclareMathOperator{\Ac}{at}
\DeclareMathOperator{\supp}{supp}
\DeclareMathOperator{\Hom}{Hom}
\DeclareMathOperator{\Pic}{Pic}
\DeclareMathOperator{\Gal}{Gal}
\DeclareMathOperator{\fc}{frac}
\DeclareMathOperator{\Ann}{Ann}
\DeclareMathOperator{\Mod}{Mod}
\DeclareMathOperator{\Cone}{Cone}
\DeclareMathOperator{\FI}{FI}
\DeclareMathOperator{\End}{End}
\DeclareMathOperator{\Alb}{Alb}
\DeclareMathOperator{\Ext}{Ext}
\DeclareMathOperator{\ab}{ab}
\DeclareMathOperator{\Jac}{Jac}
\DeclareMathOperator{\coker}{coker}
\DeclareMathOperator{\fr}{frac}
\DeclareMathOperator{\Int}{Int}
\let\Span\relax
\DeclareMathOperator{\Span}{Span}
\DeclareMathOperator{\Ran}{Ran}



%----Analysis
\newcommand{\summ}{\sum\limits}
% \newcommand{\norm}[1]{\left\lVert#1\right\rVert}
\newcommand{\thicc}{\bigg}
\newcommand{\eps}{\varepsilon}
\newcommand*\cls[1]{\overline{#1}}
\newcommand{\ind}{\mathbbm{1}}
\DeclareMathOperator{\sgn}{sgn}


%--------Theorem environments--------
\newtheorem{definition}{Definition}[]
\newtheorem{lemma}{Lemma}[]
\newtheorem{corollary}{Corollary}[]
\newtheorem{theorem}{Theorem}[]
\theoremstyle{remark}
\newtheorem*{claim}{Claim}


\newenvironment{solution}
{\begin{proof}[Solution]}
{\end{proof}}


\makeatletter
\newcommand{\thickhline}{%
    \noalign {\ifnum 0=`}\fi \hrule height 1pt
    \futurelet \reserved@a \@xhline
}
\newcolumntype{"}{@{\hskip\tabcolsep\vrule width 1pt\hskip\tabcolsep}}
\makeatother

% --------Problem environment--------
\setlength\parindent{0pt}
\setcounter{secnumdepth}{0}
\newcounter{partCounter}
\newcounter{homeworkProblemCounter}
\setcounter{homeworkProblemCounter}{1}


\newenvironment{homeworkProblem}[1][-1]{
    \ifnum#1>0
        \setcounter{homeworkProblemCounter}{#1}
    \fi
    \section{Problem \arabic{homeworkProblemCounter}}
    \setcounter{partCounter}{1}
    \stepcounter{homeworkProblemCounter}
}


%--------Metadata--------
\title{MATH 7410 Homework 1}
\author{James Harbour}

\begin{document}
\maketitle

\begin{homeworkProblem}
  Let $(X,\mu)$ be a $\sigma$-finite measure space.\\

  \textbf{(a)}: Prove that if $x,y\in\R_+$ and $0<p<1$, then $(x+y)^p\leq x^p+y^p$.

  \begin{proof}
    Fix $y\in\R_+$ and consider the function $f:[0,+\infty)\to \R$ given by $f(x) = (x+y)^p-(x^p+y^p)$. Note that $f(0)=0$. As $0<p<1$,
    \[
      x^{p-1}\geq (x+y)^{p-1}\implies 0\geq p\lr{(x+y)^{p-1}-x^{p-1}} = f'(x)
    \]
    so $f$ is nonincreasing and $f(0) = 0$ whence $f(x)\leq 0$ for all $x\in \R_+$ as desired.
  \end{proof}

  \textbf{(b)}: Fix $0<p<\infty$ and prove that $L^p(X,\mu)$ is a vector space under the natural operations of addition and scalar multiplication.

  \textbf{(c)}: Fix $0<p<1$ and define $d:L^p(X,\mu)\times L^p(X,\mu) \to [0,\infty)$ by $d(f,g) = \norm{f-g}_p^p$. Prove that $d$ is a metric and that addition and multiplication are continuous with respect to $d$. \\

\end{homeworkProblem}


\begin{homeworkProblem}
  Let $X$ be a Banach space.\\

  \textbf{(a)}: If $Y,Z$ are Banach spaces, and $S\in B(X,Y), T\in B(Y,Z)$, prove that $\norm{TS}\leq \norm{T}\norm{S}$.

  \begin{proof}
    For all $x\in X$, observe that
    \[
      \norm{TSx} \leq \norm{T}\norm{Sx}\leq \norm{T}\norm{S}\norm{x},
    \]
    so by definition $\norm{TS}\leq \norm{T}\norm{S}$.
  \end{proof}

  \textbf{(b)}: If $T\in B(X)$ and $\norm{T}<1$, prove that $1-T$ is invertible.\\

  \begin{proof}
    Define $S_n = \sum_{k=0}^{n}T^k$. Note that
    \[
      \norm{S_n} \leq \sum_{k=0}^{n}\norm{T}^k\leq \sum_{k=0}^{\infty} \norm{T}^k = (1-\norm{T})^{-1}.
    \]
    As $X$ is Banach and $S_n$ is Cauchy, there exists an $S\in B(X)$ such that $\norm{S_n-S}\xrightarrow{n\to\infty} 0$. \\

    Fix $\eps>0$. Choose $N\in \N$ such that for $n\geq N$, $\norm{T}^n<\eps/2$ and $\norm{S-S_n} < \frac{\eps}{2\norm{1-T}}$. Then for $n\geq N$,
    \begin{align*}
      \norm{S(1-T)-1} \leq \norm{(S-S_n)(1-T)}+\norm{S_n(1-T)-1} &\leq \norm{S-S_n}\norm{1-T} + \norm{S_n(1-T)-1} \\
      &< \frac{\eps}{2} + \norm{\sum_{k=0}^{n} T^k(1-T)-1} = \frac{\eps}{2}\norm{T^{k+1}}<\eps.
    \end{align*}
    As $\eps>0$ was arbitrary, $S(1-T) = 1$.
  \end{proof}

  \textbf{(c)}: IF $T\in B(X)$ is invertible and $S\in B(X)$ has $\norm{S-T}<\norm{T^{-1}}^{-1}$, then $S$ is invertible. Use this to show that the set of invertible elements $Inv(B(X))$ is open.

  \begin{proof}
    Observe that
    \[
      \norm{T^{-1}S - 1} \leq \norm{T^{-1}}\norm{S-T} < 1
    \]
    by assumption, so part (b) implies that $-T^{-1}S = 1-(T^{-1}S - 1)$ is invertible, whence $S$ is invertible as the invertible elements of $B(X)$ form a group with multiplication.\\

    Hence if $T\in Inv(B(X))$ then $B_{\norm{T^{-1}}^{-1}}(T,\norm{\cdot})\sub Inv(B(X))$, so $Inv(B(X))$ is open as it is the union of all such open balls.
  \end{proof}

\end{homeworkProblem}

\begin{homeworkProblem}
  Show that $l^\infty$ is not separable.\\

  \begin{proof}
    For each $I\in\ms{P}(\N)$, define an element $a_I\in l^{\infty}(\N)$ by $a_I(n) = \ind_{I}(n)$. Then for $I\neq J\in \ms{P}(\N)$, $B_{1/2}(a_I)\cap B_{1/2}(a_J) = \emptyset$. Thus, we have an uncountable family of pairwise disjoint balls $\{ B_{1/2}(a_I)\}_{I\in\ms{P}(\N)}$. Any dense subset of $l^{\infty}(\N)$ must intersect each of these balls, whence by disjointness this set must be uncountable. By contraposition $l^{\infty}(\N)$ is not separable.
  \end{proof}
\end{homeworkProblem}

\begin{homeworkProblem}
  Prove that if $X$ is a normed space, $M\leq X$, and both $M$ and $X/M$ are complete, then $X$ is complete.

  \begin{proof}
    Let $Q:X\to X/M$ be the natural map and $(x_n)_{n=1}^{\infty}$ a Cauchy sequence in $X$. Then by completeness, there exists a $z\in X$ such that $\norm{Q(x_n-z)}\xrightarrow{n\to\infty}0$. Choose a sequence $(m_n)_{n=1}^{\infty}$ in $M$ such that for $n\in\N$,
    \[
      \norm{Q(x_n-x)}+\frac{1}{n}\geq\norm{x_n-x-m_n}.
    \]
    Then $\norm{x_n-x-m_n}\xrightarrow{n\to\infty}0$. We claim that $(m_n)_{n=1}^{\infty}$ is Cauchy. To see this, observe that
    \begin{align*}
      \norm{m_n-m_k}&\leq \norm{x_k-x-m_k}+\norm{x_k-x-m_n} \\&\leq \norm{x_k-x-m_k}+\norm{x_n-x-m_n} + \norm{x_k-x_n} \xrightarrow{k,n\to\infty}0.
    \end{align*}
    By completeness, there is some $m\in M$ such that $m_n\to m$. Lastly, note that
    \[
      \norm{x_n-x-m} \leq \norm{x_n-x-m_n} + \norm{m_n-m} \xrightarrow{n\to\infty}0,
    \]
    so $x_n\to x + m$, whence $X$ is complete.
  \end{proof}

\end{homeworkProblem}

\begin{homeworkProblem}
  Let $\H$ be a Hilbert space and suppose $\M\leq \H$. Show that if $Q:\H\to\H/\M$ is the natural map, then $Q: \M^\perp\to \H/\M$ is an isometric isomorphism.

  \begin{proof}
    Let $f\in \H$. There are unique $f^\parallel\in \M$ and $f^\perp\in \M^\perp$ such that $f = f^\parallel+f^\perp$, whence $Q(f^\perp) = f^\perp +\M = f^\perp +f^\parallel + \M = f+ \M$. Thus $Q\vert_{\M^\perp}$ is surjective. \\

    Moreover, note that $f^\parallel$ is such that
    \[
      \norm{f^\perp} = \norm{f-f^{\parallel}} = dist(f, \M) = \inf\{\norm{f+m}:m\in M\} = \norm{Q(f^\perp)}
    \]
    so $Q\vert_{\M^\perp}$ is isometric and thus injective. As $Q$ is bounded and $\M^\perp$ is a closed linear subspace of $\H$, $Q\vert_{\M^\perp}$ is continuous. Thus $Q\vert_{\M^\perp}$ is a continuous bijection of Banach spaces, so by the Inverse mapping theorem $(Q\vert_{\M^\perp})^{-1}$ is continuous.
  \end{proof}
\end{homeworkProblem}

\end{document}
