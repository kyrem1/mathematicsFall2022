\documentclass[12pt,letterpaper]{article}

%--------Packages--------
\usepackage{amsmath, amsthm, amssymb}
\usepackage{xspace}
\usepackage{graphicx}
\usepackage{hhline}
\usepackage{amssymb}
\usepackage{array}
\usepackage{braket}
\usepackage{multicol}
\usepackage{mathtools}
\usepackage{enumerate}
\usepackage{delarray}
\usepackage{mathtools}
\usepackage{fullpage}
\usepackage{faktor} % For quotients
\usepackage{mathrsfs}

\usepackage[italicdiff]{physics} % For differentials
\usepackage{bbm} % For indicator

% \usepackage{quiver}
\usepackage[linguistics]{forest}




%--------Page Setup--------

\pagestyle{empty}%

\setlength{\hoffset}{-1.54cm}
\setlength{\voffset}{-1.54cm}

\setlength{\topmargin}{0pt}
\setlength{\headsep}{0pt}
\setlength{\headheight}{0pt}

\setlength{\oddsidemargin}{0pt}

\setlength{\textwidth}{195mm}
\setlength{\textheight}{250mm}


%--------Macros--------

\newcommand{\sub}{\subseteq}
\newcommand{\lcm}{\text{lcm}}
\newcommand{\mc}[1]{\mathcal{#1}}
\newcommand{\mf}[1]{\mathfrak{#1}}
\newcommand{\ms}[1]{\mathscr{#1}}
\newcommand{\sO}{\mathcal{O}}
\newcommand{\cyclic}[1]{\langle#1\rangle}
\newcommand{\units}[1]{#1 ^{\times}}
\newcommand{\la}{\langle}
\newcommand{\ra}{\rangle}
\newcommand{\lr}[1]{\left(#1\right)}
\newcommand{\lrvert}[1]{\left\lvert#1\right\rvert}

\DeclarePairedDelimiterX{\inp}[2]{\langle}{\rangle}{#1, #2}

%----Switch phi and varphi
% \let\temp\phi
% \let\phi\varphi
% \let\varphi\temp

\newcommand{\C}{\mathbb{C}}
\newcommand{\F}{\mathbb{F}}
\newcommand{\E}{\mathbb{E}}
\newcommand{\N}{\mathbb{N}\xspace}
\newcommand{\I}{\mathbb{I}\xspace}
\newcommand{\R}{\mathbb{R}\xspace}
\newcommand{\Z}{\mathbb{Z}\xspace}
\newcommand{\Q}{\mathbb{Q}\xspace}
\newcommand{\G}{\mathbb{G}\xspace}

\renewcommand{\H}{\mathcal{H}}
\newcommand{\M}{\mathcal{M}}

\DeclareMathOperator{\Spec}{Spec}
\DeclareMathOperator{\res}{res}
% \DeclareMathOperator{\Tr}{Tr}
\DeclareMathOperator{\ord}{ord}
\DeclareMathOperator{\Sym}{Sym}
% \DeclareMathOperator{\dv}{div}
\DeclareMathOperator{\alb}{alb}
\DeclareMathOperator{\img}{Im}
\DeclareMathOperator{\et}{et}
\DeclareMathOperator{\ck}{coker}
\DeclareMathOperator{\Reg}{Reg}
\DeclareMathOperator{\Cor}{Cor}
\DeclareMathOperator{\Ac}{at}
\DeclareMathOperator{\supp}{supp}
\DeclareMathOperator{\Hom}{Hom}
\DeclareMathOperator{\Pic}{Pic}
\DeclareMathOperator{\Gal}{Gal}
\DeclareMathOperator{\fc}{frac}
\DeclareMathOperator{\Ann}{Ann}
\DeclareMathOperator{\Mod}{Mod}
\DeclareMathOperator{\Cone}{Cone}
\DeclareMathOperator{\FI}{FI}
\DeclareMathOperator{\End}{End}
\DeclareMathOperator{\Alb}{Alb}
\DeclareMathOperator{\Ext}{Ext}
\DeclareMathOperator{\ab}{ab}
\DeclareMathOperator{\Jac}{Jac}
\DeclareMathOperator{\coker}{coker}
\DeclareMathOperator{\fr}{frac}
\DeclareMathOperator{\Int}{Int}
\let\Span\relax
\DeclareMathOperator{\Span}{Span}
\DeclareMathOperator{\Ran}{Ran}
\DeclareMathOperator{\ran}{ran}


%----Analysis
\newcommand{\summ}{\sum\limits}
% \newcommand{\norm}[1]{\left\lVert#1\right\rVert}
\newcommand{\thicc}{\bigg}
\newcommand{\eps}{\varepsilon}
\newcommand*\cls[1]{\overline{#1}}
\newcommand{\ind}{\mathbbm{1}}
\DeclareMathOperator{\sgn}{sgn}
\DeclareMathOperator{\Prob}{Prob}

%--------Theorem environments--------
\newtheorem{definition}{Definition}[]
\newtheorem{lemma}{Lemma}[]
\newtheorem{corollary}{Corollary}[]
\newtheorem{theorem}{Theorem}[]
\theoremstyle{remark}
\newtheorem*{claim}{Claim}


\newenvironment{solution}
{\begin{proof}[Solution]}
{\end{proof}}


\makeatletter
\newcommand{\thickhline}{%
    \noalign {\ifnum 0=`}\fi \hrule height 1pt
    \futurelet \reserved@a \@xhline
}
\newcolumntype{"}{@{\hskip\tabcolsep\vrule width 1pt\hskip\tabcolsep}}
\makeatother

% --------Problem environment--------
\setlength\parindent{0pt}
\setcounter{secnumdepth}{0}
\newcounter{partCounter}
\newcounter{homeworkProblemCounter}
\setcounter{homeworkProblemCounter}{1}


\newenvironment{homeworkProblem}[1][-1]{
    \ifnum#1>0
        \setcounter{homeworkProblemCounter}{#1}
    \fi
    \section{Problem \arabic{homeworkProblemCounter}}
    \setcounter{partCounter}{1}
    \stepcounter{homeworkProblemCounter}
}


%--------Metadata--------
\title{MATH 7410 Homework 4 (In-Progress)}
\author{James Harbour}

\begin{document}
\maketitle

\begin{homeworkProblem}
    Let $ X $ be a normed space and $ x_{n} $ a sequence in $ X $ such that $ x_{n}\to x $ weakly. Show that there is a sequence $ y_{n} $ such that $ y_{n}\in co\{x_{1},\ldots,x_{n}\} $ and $ \norm{y_{n}-x}\to 0. $

    \begin{proof}
        Let $ C_{n} = co\{x_{1},\ldots,x_{n}\} $ and $ C = \bigcup_{n=1}^{\infty}C_{n} $. Note that $ C = co\{x_{i}: i\in \N\} $ is convex. Thus $ x\in \cls{C}^{wk} = \cls{C}^{\norm{\cdot}} $, whence there is some sequence $ (z_{n})_{n=1}^{\infty} $ in $ C $ such that $ \norm{z_{n} - x} \to 0 $. Let $ k_{n}\in \N $ be such that $ z_{n}\in C_{k_{n}} $. \\

        Note that a sequence $ a_{n} $ in $ X $ converges to $ 0 $ in norm if and only if for every $ \eps > 0 $, the set $ \{n\in\N: \norm{a_{n}} \geq \eps\} $ is finite. This condition is invariant under rearrangements, so without loss of generality we may take the sequence $ (k_{n})_{n\in\N} $ to be nondecreasing. Construct a new sequence $ (y_{m})_{m\in\N} $ as follows. For $ m < k_{1} $, set $ y_{m} = 0 $. For $ n\in\N $ and $ k_{n}\leq m < k_{n+1} $, set $ y_{m} = z_{n} $. \\

        The sequence $ y_{m}-x $ still has the condition that for all $ \eps>0 $ the set of indices whose corresponding elements have norm at least $ \eps $  is finite, as we have only added a finite number of elements to this set. Thus $ y_{n}\in C_{n} $ for all $ n\in \N $ and $ \norm{y_{n}-x}\to 0 $.
    \end{proof}
\end{homeworkProblem}



\begin{homeworkProblem}
    If $ \mc{H} $ is a Hilbert space and $ h_{n} $ is a sequence in $ \mc{H} $ such that $ h_{n}\to h $ weakly and $ \norm{h_{n}}\to \norm{h} $, show that $ \norm{h_{n}-h}\to 0 $.

    \begin{proof}
        
    \end{proof}
\end{homeworkProblem}



\begin{homeworkProblem}
    If $ X $, $ Y $ are Banach spaces and $ B\in B(Y^{*},X^{*}) $, then $ B = A^{*} $ for some $ A\in B(X,Y) $ if and only if $ B $ is $ wk^* $-continuous.

    \begin{proof}\ \\
        \underline{($\implies$)}: Suppose that $ B = A^{*} $ for some $ A\in B(X,Y) $ and let $ (\psi_{\alpha})_{\alpha\in I} $ be a net in $ Y^{*} $ such that $ \psi_{\alpha}\to \psi\in Y^{*} $ $ \text{weak}^{*}$. Fix $ x\in X $. Then
        \[
            B(\psi_{\alpha})(x) = A^{*}(\psi_{\alpha})(x) = \psi_{\alpha}(Ax) \xrightarrow{\alpha\in I} \psi(Ax) = B(\psi)(x),
        \]
        so $ B(\psi_{\alpha})\to B(\psi) $ $ \text{weak}^{*} $, i.e. $ B $ is $ \text{weak}^{*} $-continuous.\\

        \underline{($\impliedby$)}: Suppose that $ B $ is $ wk^{*} $-continuous. Let $ \iota_{X},\iota_{Y} $ be the canonical injections into the corresponding double-duals. For shorthand, we may write $ \hat{x}:=\iota_{X}(x) $ and similarly for $Y$.
        Noting that $ B^{*}\in B(X^{**},Y^{**}) $ we investigate what occurs when $ B^{*} $ is restricted to the image of $ X $ inside its double dual. \\

        Fix $ x\in X $. We claim that $ B^{*}(\hat{x})\in (Y^{*},wk^{*})^{*} $. To this end, let $ (\phi_{\alpha})_{\alpha\in I} $ be net in $ Y^{*} $ such that $ \phi_{\alpha}\to \phi\in Y^{*} $ $\text{weak}^{*}$. Then,
        \[
            B^{*}(\hat{x})(\phi_{\alpha}) = \hat{x}(B(\phi_{\alpha})) = B(\phi_{\alpha})(x)\to B(\phi)(x) = B^{*}(\hat{x})(\phi),
        \]
        so $ B^{*}(\hat{x}) $ is $ \text{weak}^{*} $-continuous, whence there exists some $ y\in Y $ such that $ B^{*}(\hat{x}) = \hat{y} $. 

        %TODO uniqueness of y (or independence upon choice of y), show A is bounded operator. Show B = A^*
    
    \end{proof}

\end{homeworkProblem}


\begin{homeworkProblem}
    Let $ X, Y $ be Banach spaces over $ \F\in \{\C,\R\} $. For $ C\sub B(X,Y) $ convex and $ F\sub X $ finite, set $ C_{F} = \{(Tx)_{x\in F}:T\in C\}\sub Y^{\oplus F} $. Equip $ Y^{\oplus F} $ with the norm 
    \[
        \norm{(y_{x})_{x\in F}} = \sum_{x\in F} \norm{y_{x}}.
    \]
    \textbf{(a)}: Let $ C\sub B(X,Y) $ be convex. Show that $ T\in \cls{C}^{SOT} $ if and only if for every $ F\sub X $ finite, we have that $ (Tx)_{x\in F}\in \cls{C_{F}}^{\norm{\cdot}} $. Show that $ T\in \cls{C}^{WOT} $ if and only if for every $ F\sub X $ finite, we have that $ (Tx)_{x\in F}\in \cls{C_{F}}^{weak}$. 

    \begin{proof}\ \\
        \underline{$ \implies $}: Suppose $ T\in \cls{C}^{SOT} $, so there is some net $ (T_{\alpha})_{\alpha\in I} $ in $ C $ such that $ T_{\alpha}\to T $ SOT. Let $ F\sub X $ finite. Then $ \norm{T_{\alpha}x - Tx}\to 0 $ for every $ x\in F $. Since $ F $ is finite,
        \[
            \norm{(T_{\alpha}x)_{x\in F}-(Tx)_{x\in F}} = \sum_{x\in F}\norm{T_{\alpha}x-Tx} \xrightarrow{\alpha\in I} 0
        \]
        so $ T\in \cls{C}^{SOT} $. \\

        \underline{$ \impliedby $}:


    \end{proof}

    \textbf{(b)}: Suppose that $ C\sub B(X,Y) $ is convex. Show that $ \cls{C}^{WOT} = \cls{C}^{SOT} $. \\

    \textbf{(c)}: IF $ \phi:B(X,Y)\to \F $ is linear, show that $ \phi $ is WOT-continuous if and only if it is SOT-continuous.


\end{homeworkProblem}



\begin{homeworkProblem}
    Let $ I $ be a set and $ \mc{M} $ be the set of all $ m\in l^{\infty}(I)^{*} $ such that :
    \begin{itemize}
        \item $ m(f) \geq 0 $ for all $ f\geq 0 $,
        \item $ m(1) = 1 $.
    \end{itemize}
    Identify $ \Prob(I) $ with $ \{f\in l^{1}(I): f\geq 0, \norm{f}_{1}= 1\} $ and view $ l^{1}(I) \sub l^{\infty}(I)^{*} $ by $ f\mapsto \phi_{f} $ where $ \phi_{f}(g) = \sum_{i\in I} f(i) g(i) $. Show that $ \Prob(I) $ is $ \text{weak}^{*} $-dense in $ \mc{M} $. 

    \begin{proof}
        Suppose, for the sake of contradiction, that there is some $ \psi\in \mc{M}\setminus\cls{\Prob(I)}^{wk^{*}} $.
        Then there exists some $ wk^{*} $-continuous linear functional $T:l^{\infty}(I)^{*}\to\F $ such that $ \ker(T)\supseteq \cls{\Prob(I)}^{wk^{*}} $ and $ T(\psi) = 1 $.
        Note that $ T\in (l^{\infty}(I)^{*}, wk^{*})^{*} = l^{\infty}(I)$, so there is some $ f\in l^{\infty}(I) $ such that $ T = \hat{f} $, i.e. $ T(\phi) = \phi(f) $ for all $ \phi\in l^{\infty}(I)^{*}$.
        Then, for $ i\in I $
        \[
            0 = T(\delta_{i}) = \delta_{i}(f) = f(i),
        \]
        whence $ f=0 $ and consequently $ T = 0 $, contradicting that $ T(\psi) = 1 $.

    \end{proof}
\end{homeworkProblem}



\end{document}
